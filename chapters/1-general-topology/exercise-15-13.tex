\subsection{}

There is a typo in the book:
instead of diagonal matrices, we want to consider \emph{triangular} matrices.%
\footnote{
	The group of diagonal matrices is abelian, whence each of its subsets is normal.
}
We thus consider the subgroup~$G$ of~$\GL(2, ℝ)$ given by
\[
	G
	≔
	\left\{
		\begin{bmatrix}
			a & b   \\
			0 & 1/a
		\end{bmatrix}
	\suchthat*
		a ∈ ℝ^×,
		b ∈ ℝ
	\right\}
\]

Let~$N$ be a normal neighbourhood of the identity.
This neighbourhood contains a matrix of the form
\[
	T
	=
	\begin{bmatrix}
		1 & b \\
		0 & 1
	\end{bmatrix}
\]
with~$b > 0$.
(We may choose~$b$ sufficiently small.)
It follows for the matrix
\[
	D
	=
	\begin{bmatrix}
		a & 0   \\
		0 & 1/a
	\end{bmatrix}
\]
with~$a > 1$ that~$N$ also contains the matrix
\[
	D T D^{-1}
	=
	\begin{bmatrix}
		1 & a^2 b \\
		0 & 1
	\end{bmatrix} \,,
\]
and inductively the matrix
\[
	D^n T D^{-n}
	=
	\begin{bmatrix}
		1 & a^{2n} b \\
		0 & 1
	\end{bmatrix}
\]
for every~$n ≥ 0$.
We have~$a^n b \to ∞$ as~$n \to ∞$, so we find that~$N$ is unbounded.

This entails that no bounded neighbourhood of the identity (e.g., no open ball) contains a normal neighbourhood of the identity.
