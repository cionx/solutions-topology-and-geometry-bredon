\subsection{}



\subsubsection*{Existence}

If~$X$ itself is already irreducible, then there is nothing left to do.
Otherwise, there exist proper, closed subsets~$C_1$ and~$C_2$ of~$X$ with~$X = C_1 ∪ C_2$:
\[
	\begin{tikzcd}[every arrow/.append style={dash}, column sep = small]
		{}
		&
		X
		\arrow{dl}
		\arrow{dr}
		&
		{}
		\\
		C_1
		&
		{}
		&
		C_2
	\end{tikzcd}
\]
If both~$C_1$ and~$C_2$ are irreducible, then we are finished.
Otherwise, suppose for example that~$C_1$ is reducible.
There then exist proper, closed subsets~$C'_1$ and~$C'_2$ of~$C_1$ with~$C_1 = C'_1 ∪ C'_2$:
\[
	\begin{tikzcd}[every arrow/.append style={dash}, column sep = small]
		{}
		&
		{}
		&
		X
		\arrow{dl}
		\arrow{dr}
		&
		{}
		\\
		{}
		&
		C_1
		\arrow{dl}
		\arrow{dr}
		&
		{}
		&
		C_2
		\\
		C'_1
		&
		{}
		&
		C'_2
		&
		{}
	\end{tikzcd}
\]
We note that~$C'_1$ and~$C'_2$ are already closed in~$X$.
If~$C'_1$,~$C'_2$ and~$C_2$ are irreducible are finished.
Otherwise, we can continue the above procedure, enlarging the above binary tree by two nodes each step.

Note that because~$X$ is a Zariski space, this binary tree cannot grow indefinitely:
Otherwise we’d have an infinite, strictly increasing sequence of binary trees.
The union of these trees is then a binary tree of infinite height, and such a tree always admits an infinite downwards path.%
\footnote{
	Indeed, beginning from the root of the tree, one of its two branches must again have infinite height.
	We can then go into this branch and proceed by induction.
}
Such a path would be an infinite strictly decreasing sequence of closed subsets of~$X$.

Thus, after finitely many steps, we arrive at a binary tree
\[
	\begin{tikzcd}[every arrow/.append style={dash}, column sep = small]
		{}
		&
		{}
		&
		{}
		&
		X
		\arrow{dll}
		\arrow{drr}
		&
		{}
		&
		{}
		&
		{}
		\\
		{}
		&
		C_1
		\arrow{dl}[description]{⋰}
		\arrow{dr}[description]{⋱}
		&
		{}
		&
		{}
		&
		{}
		&
		C_2
		\arrow{dl}[description]{⋰}
		\arrow{dr}[description]{⋱}
		&
		{}
		\\
		C^{(n)}_1
		&
		{}
		&
		C^{(n')}_2
		&
		{}
		&
		C^{(n'')}_1
		&
		{}
		&
		C^{(n''')}_2
	\end{tikzcd}
\]
in which all leaves are closed, irreducible subsets of~$X$, whose union is all of~$X$.

We have thus arrived at a decomposition
\[
	X = Y_1 ∪ \dotsb ∪ Y_n
\]
into closed, irreducible subspaces~$Y_1, \dotsc, Y_n$ of~$X$.
If now~$Y_i ⊆ Y_j$ for some indices~$i ≠ j$, then we can simply eliminate~$Y_i$ from this decomposition.
After finitely many such eliminations we can assume that~$Y_i ⊈ Y_j$ whenever~$i ≠ j$.



\subsubsection*{Uniqueness}

We observe that every irreducible subspace~$Y$ of~$X$ is contained in some~$Y_i$.
Indeed, we have the induced decomposition
\[
	Y
	=
	Y ∩ X
	=
	Y ∩ (Y_1 ∪ \dotsb ∪ Y_n)
	=
	(Y ∩ Y_1) ∪ \dotsb ∪ (Y ∪ Y_n) \,,
\]
with~$Y ∩ Y_i$ closed in~$Y$ for every index~$i$.
It follows from the irreducibility of~$Y$ that~$Y = Y ∩ Y_i$ for some index~$i$, which means that~$Y ⊆ Y_i$.

Suppose now that we are given another decomposition~$X = Z_1 ∪ \dotsb ∪ Z_m$ into closed, irreducible subspaces such that~$Z_i ⊈ Z_j$ whenever~$i ≠ j$.
It follows from the above observation that there exists for every index~$i = 1, \dotsc, n$ some index~$σ(i)$ with~$Y_i ⊆ Z_{σ(i)}$.
Similarly, there exists for every index~$j = 1, \dotsc, m$ some index~$τ(j)$ with~$Z_j ⊆ Y_{τ(j)}$.
We have for every index~$i = 1, \dotsc, n$ the inclusions
\begin{equation}
	\label{nesting of irreducible components}
	Y_i ⊆ Z_{σ(i)} ⊆ Y_{τ(σ(i))} \,,
\end{equation}
and therefore the equality~$i = τ(σ(i))$.
We hence have~$Y_i ⊆ Z_{σ(i)} ⊆ Y_i$, and thus~$Y_i = Z_{σ(i)}$.

We find similarly that~$σ(τ(j)) = j$ for every index~$j = 1, \dotsc, m$.
Together this shows that the two maps
\[
	σ \colon \{ 1, \dotsc, n \} \to \{ 1, \dotsc, m \}\,,
	\qquad
	τ \colon \{ 1, \dotsc, m \} \to \{ 1, \dotsc, n \}\,,
\]
are mutually inverse.
This implies that~$n = m$ and that~$σ$ is a permutation.
Together with the equalities~$Y_i = Z_{σ(i)}$ for~$i = 1, \dotsc, n$ this altogether shows that the family~$Z_1, \dotsc, Z_m$ is simply a permutation of the family~$Y_1, \dotsc, Y_n$.
