\section{}

It should be noted that~$A$ needs to be nonempty.
Otherwise,~$d(x) = ∞$ for every~$x ∈ X$.
So in the following, let~$A$ be nonempty.

Let~$x_1$ and~$x_2$ be two arbitrary points in~$X$.
We have for every point~$y$ of~$A$ the inequality
\[
	\dist(x_1, y) ≤ \dist(x_1, x_2) + \dist(x_2, y)
\]
and therefore the sequence of (in)equalities
\begin{align*}
	d(x_1)
	&=
	\inf_{y ∈ A} \dist(x_1, y) \\
	&≤
	\inf_{y ∈ A} {} \bigl( \dist(x_1, x_2) + \dist(x_2, y) \bigr) \\
	&=
	\dist(x_1, x_2) + \inf_{y ∈ A} \dist(x_2, y) \\
	&=
	\dist(x_1, x_2) + d(x_2) \,.
\end{align*}
This shows that
\[
	d(x_1) - d(x_2) ≤ \dist(x_1, x_2)
\]
By switching around the roles of~$x_1$ and~$x_2$ we also find that
\[
	-(d(x_1) - d(x_2))
	=
	d(x_2) - d(x_1)
	≤
	\dist(x_2, x_1)
	=
	\dist(x_1, x_2) \,,
\]
and thus altogether
\[
	\abs{d(x_1) - d(x_2)}
	≤
	\dist(x_1, x_2) \,.
\]
This shows that the function~$d$ is Lipschitz continuous (with Lipschitz constant~$1$), and thus continuous.
