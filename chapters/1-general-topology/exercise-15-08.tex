\subsection{}



\subsubsection*{$\SOrt(n)$ is connected for every~$n ≥ 0$}

\begin{lemma}
	The sphere~$\sphere^n$ is arc connected for every~$n ≥ 1$.
\end{lemma}

\begin{proof}
	The interval~$[0, 1]$ is arc connected because it is convex.
	The continuous map~$[0, 1] \to \sphere^1$ given by~$φ \mapsto \eul^{2 \pi \img φ}$ is surjective, whence~$\sphere^1$ is arc connected.

	Suppose that~$\sphere^n$ is arc connected for some~$n$.
	The space~$\sphere^n × [-1, 1]$ is then again arc connected, and the map
	\[
		\sphere^n × [-1, 1] \to \sphere^{n + 1} \,,
		\quad
		(x, h) \mapsto \Bigl( \sqrt{1 - h^2} ⋅ x, h \Bigr)
	\]
	is continuous and surjective.
	It follows that~$\sphere^{n + 1}$ is also connected.
\end{proof}

The groups~$\SOrt(0)$ and~$\SOrt(1)$ are trivial and therefore connected.

Let now~$n ≥ 2$ and suppose we have already proven that~$\SOrt(n - 1)$ is connected.
The quotient~$\SOrt(n) / \SOrt(n - 1)$ is homeomorphic to~$\sphere^{n - 1}$, therefore arc connected, and thus connected.
It follows from Exercise~I.15.5 that~$\SOrt(n)$ is connected.



\subsubsection*{$\SOrt(n)$ is the connected component of the identity in~$\Ort(n)$}

If~$n = 0$ then both~$\Ort(n)$ and~$\SOrt(n)$ are trivial, whence~$\SOrt(n)$ is the connected component of the identity.
Let in the following~$n ≥ 1$.

Let~$G ≔ \Ort(n)$, let~$G_0$ be its component of the identity, and let~$H ≔ \SOrt(n)$.
We know that~$H$ is connected and contains the identity, whence~$H ⊆ G_0 ⊆ G$.
The inclusion~$H ⊆ G_0$ implies that~$[G : H] ≥ [G : G_0]$.%
\footnote{
	Given a group~$G$ and subgroup~$H$ of~$G$, we denote by~$[G : H]$ the index of~$H$ in~$G$.
}

The determinant function is a continuous, surjective homomorphism of groups from~$G$ to~$\{ 1, -1 \}$.
If~$G$ were connected then so would be~$\{ 1, -1 \}$, which is not the case.
We hence see that~$G$ is not connected, so that~$[G : G_0] ≥ 2$.
But we also have~$[G : H] = 2$ because~$H$ is precisely the kernel of the determinant function.
Therefore,~$[G : H] = 2 ≤ [G : G_0]$.

We have altogether shown that~$[G : G_0] = [G : H] = 2$.
This equality of finite indices implies that the inclusion~$H ⊆ G_0$ is already an equality.
