\subsection{}

Let~$Ω$ denotes the least uncountable ordinal, and let~$X = Ω ∪ \{ Ω \}$ denotes its successor ordinal.
We make~$X$ into a topological space via the order topology, i.e., the topology generated by all one-sided intervals~$L(α) ≔ \{ β ∈ X \suchthat β < α \}$ together with all one-sided intervals~$U(α) ≔ \{ β ∈ X \suchthat β > α \}$ for~$α ∈ X$.

We note that no~$L(α)$ contains~$Ω$ as an element, because~$Ω$ is the maximal element of~$X$.



\subsubsection*{Idea and setup}

We know that metric spaces are first-countable.
So to show that the topology on~$X$ is not induced by any metric, we show that~$X$ is not first-countable.
To this end, we show that~$Ω$ does not admit a countable neighbourhood basis.

Suppose otherwise that~$Ω$ admits a countable neighbourhood basis
\[
	N_1, N_2, N_3, \dotsc
\]
There exists for every index~$i$ some basis set~$B_i$ with~$N_i ⊇ B_i ∋ Ω$.
The sets~$B_1, B_2, B_3, \dotsc$ are again a countable neighbourhood basis of~$Ω$.



\subsubsection*{Step~1}

We first want to understand how a basis set that contains~$Ω$ as an element looks like.

We have already noted that the only subbasis sets that contain~$Ω$ as an element are the one-sided intervals~$U(α)$ for~$α ∈ Ω$.

Every basis element~$B$ is a finite intersection~$S_1 ∩ \dotsb ∩ S_n$ of subbasis sets~$S_i$, and~$Ω$ is an element of~$B$ if and only if it is simultaneously an element of each~$S_i$.
This means that each~$S_i$ is of the form~$S_i = U(α_i)$ for some~$α_i ∈ Ω$.
Then either~$n = 0$ and thus~$B = X$, or
\[
	B
	=
	S_1 ∩ \dotsb ∩ S_n
	=
	U(α_1) ∩ \dotsb ∩ U(α_n)
	=
	U(\max(α_1, \dotsc, α_n)) \,.
\]
This shows that the basis sets containing~$Ω$ as an element are precisely~$X$ and~$U(α)$ with~$α ∈ Ω$.



\subsubsection*{Step~2}

We observe that the intersection of all neighbourhoods of~$Ω$ in~$X$ is~$\{ Ω \}$.
Indeed, we have
\begin{align*}
	{}&
	⋂ {} \{ N ⊆ X \suchthat \text{$N$ is a neighbourhood of~$Ω$} \}
	\\
	⊇{}&
	⋂_{α ∈ Ω} {} \{ β ∈ X \suchthat β > α \}
	\\
	⊇{}&
	⋂_{α ∈ X - \{ Ω \}} {} \{ β ∈ X \suchthat β > α \}
	\\
	={}&
	\{ Ω \} \,,
\end{align*}
because~$Ω$ is the maximal element of~$X$.
But for each such neighbourhood~$N$ there exists some index~$i$ with~$N ⊇ B_i ∋ Ω$, whence we also find that
\[
	⋂ {} \{ N ⊆ X \suchthat \text{$N$ is a neighbourhood of~$Ω$} \}
	=
	⋂_{i ≥ 1} B_i \,.
\]
Together, this shows that~$⋂_{i ≥ 1} B_i = \{ Ω \}$.



\subsubsection*{Step~3}

Each~$B_i$ is either~$X$ or of the form~$U(α)$ for some~$α ∈ Ω$.
There hence exist countable many~$α_1, α_2, \dotsc ∈ Ω$ with
\[
	\{ Ω \} = ⋂_{i ≥ 1} U(α_i)
\]
as subsets of~$X$.
Taking complements, this means that
\begin{align*}
	Ω
	=
	X - \{ Ω \}
	=
	X - ⋂_{i ≥ 1} {} U(α_i)
	&=
	⋃_{i ≥ 1} {} \bigl( X - U(α_i) \bigr) \\
	&=
	⋃_{i ≥ 1} {} \{ β ∈ Ω \suchthat β ≤ α_i \}
	=
	⋃_{i ≥ 1} α_i \,.
\end{align*}
The ordinal~$Ω$ is thus the union of its smaller ordinals~$α_i$.
However,~$Ω$ is the \emph{least} uncountable cardinal, whence each~$α_i$ is countable.
The countable union~$⋃_{i ≥ 0} α_i$ is thus again countable.
But~$Ω$ is uncountable -- a contradiction!
