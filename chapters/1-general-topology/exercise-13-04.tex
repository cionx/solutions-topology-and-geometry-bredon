\subsection{}

We denote the described spaces by~$X_1$,~$X_2$,~$X_3$ and~$X_4$ respectively.



\subsubsection*{The homeomorphism~$X_2 ≅ X_3$}

We consider the map
\[
	f
	\colon
	[0, 1]^2 \to X_3 \,,
	\quad
	(t, u) \mapsto (\eul^{2 π i t}, \eul^{2 π i u}) \,.
\]
This map is continuous and surjective, and induces a continuous bijection
\[
	f'
	\colon
	X_2 \to X_3 \,,
	\quad
	\class{(t, u)} \mapsto (\eul^{2 π i t}, \eul^{2 π i u}) \,.
\]
The quotient space~$X_2$ is again compact, and~$X_3$ is a Hausdorff space because the circle~$\sphere^1$ is a Hausdorff space.
It follows that~$f'$ is a homeomorphism.



\subsubsection*{The homeomorphism~$X_1 ≅ X_3$}

The inclusion from~$[0, 1]^2$ into~$ℝ^2$ induces a continuous bijection from~$X_2$ to~$X_3$.
As~$X_2$ is compact, it follows that~$X_3$ is compact.
We can therefore proceed as for the homeomorphism~$X_2 ≅ X_3$.



\subsubsection*{The homeomorphism~$X_2 ≅ X_4$}

We position the anchor ring so that its center lies in the origin of~$ℝ^3$, and its rotation axis is the~\axis{$z$}.
We first construct a parametrization of~$X_4$.
One circular slice of~$X_4$ is given by
\[
	\left\{
		\vect{0 \\ R \\ 0} + r \vect{0 \\ \cos(2 π t) \\ \sin(2 π t)}
	\suchthat*
		t ∈ [0, 1]
	\right\} \,.
\]
(This set is the slice for~$x = 0$ and~$y > 0$).
Rotation around the~\axis{$z$} is given by the rotation matrices
\[
	\begin{bmatrix}
		\cos(α) &           -\sin(α) &   \\
		\sin(α) & \phantom{-}\cos(α) &   \\
		        &                    & 1
	\end{bmatrix}
\]
with~$α ∈ ℝ$.
It follows that~$X_4$ is the image of the map
\[
	f
	\colon
	[0, 1]^2 \to ℝ^3 \,,
\]
given by
\begin{align*}
	f(t, u)
	≔{}&
	\begin{bmatrix}
		\cos(2 π t) &           -\sin(2 π t) & 0 \\
		\sin(2 π t) & \phantom{-}\cos(2 π t) & 0 \\
		0           & \phantom{-}0           & 1
	\end{bmatrix}
	\vect{0 \\ R + r \cos(2 π u) \\ r \sin(2 π u)}
	\\
	={}&
	\vect{
		- \sin(2 π t) (R + r \cos(2 π u)) \\
		\cos(2 π t) (R + r \cos(2 π u)) \\
		r \sin(2 π u)
	} \,.
\end{align*}
The map~$f$ is surjective and continuous, and induces a continuous bijection from~$X_2$ to~$X_4$.
The space~$X_2$ is compact and the space~$X_4$ is a Hausdorff space, whence this continuous bijection is a homeomorphism.
