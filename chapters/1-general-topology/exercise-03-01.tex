\subsection{}



\subsubsection{}

The given set can be rewritten as
\[
	⋃ {} \{ U ⊆ A \suchthat \text{$U$ is open in~$X$} \} ≕ I \,.
\]
The set~$I$ is a union of open subsets of~$X$, and therefore again open in~$X$.
It is also a subset of~$A$, as each~$U$ is a subset of~$A$.
By construction,~$I$ is larger or equal to every other open subset of~$X$ that is contained in~$A$.
The set~$I$ is hence the largest subset of~$A$ that is open in~$X$, i.e., it is the interior of~$A$.

We can show dually that the closure of~$A$ is given by
\[
	\closure{A}
	=
	⋂ {} \{ C ⊆ X \suchthat \text{$A ⊆ C$ and~$C$ is closed in~$X$} \} \,.
\]
It follows for every point~$x$ in~$X$ that
\begin{align*}
	{}&
	x \notin \closure{A}
	\\
	\iff{}&
	\text{there exists~$C ⊆ X$ such that~$C$ is closed,~$A ⊆ C$, and~$x ∉ C$}
	\\
	\iff{}&
	\text{there exists~$U ⊆ X$ such that~$U$ is open,~$A ⊆ X - U$, and~$x ∉ X - U$}
	\\
	\iff{}&
	\text{there exists~$U ⊆ X$ such that~$U$ is open,~$A ∩ U = ∅$, and~$x ∈ U$}
	\\
	\iff{}&
	x ∉ \{ a ∈ X \suchthat \text{if~$U ⊆ X$ is open with~$A ∩ U = ∅$, then~$a ∉ U$} \} \,,
\end{align*}
and therefore
\[
	\closure{A}
	=
	\{ a ∈ X \suchthat \text{if~$U ⊆ X$ is open with~$A ∩ U = ∅$, then~$a ∉ U$} \} \,.
\]



\subsubsection{}

These equivalences follow directly from the fact that~$\int(A)$ is the largest open subset of~$X$ contained in~$A$, and that~$\closure{A}$ is the smallest closed subset of~$X$ containing~$A$.



\subsubsection{}

We have the sequence of equalities
\begin{align*}
	{}&
	X - \int(A)
	\\
	={}&
	X - ⋃ {} \{ U ⊆ X \suchthat \text{$U$ is open and~$U ⊆ A$} \}
	\\
	={}&
	⋂ {} \{ X - U \suchthat \text{$U ⊆ X$ is open and~$U ⊆ A$} \}
	\\
	={}&
	⋂ {} \{ C \suchthat \text{$C ⊆ X$ is closed and~$X - C ⊆ A$} \}
	\\
	={}&
	⋂ {} \{ C \suchthat \text{$C ⊆ X$ is closed and~$X - A ⊆ C$} \}
	\\
	={}&
	\closure{ X - A } \,.
\end{align*}
It follows that
\[
	\int(X - A)
	=
	X - (X - \int(X - A))
	=
	X - \closure{X - (X - A)}
	=
	X - \closure{A} \,.
\]



\subsubsection{}

The set~$\int(A ∩ B)$ is open and contained in~$A ∩ B$, and therefore contained in both~$A$ and~$B$.
It is therefore contained in both~$\int(A)$ and~$\int(B)$, and thus in~$\int(A) ∩ \int(B)$.
This shows that
\[
	\int(A ∩ B) ⊆ \int(A) ∩ \int(B) \,.
\]

Conversely, the intersection~$\int(A) ∩ \int(B)$ is again open in~$X$ and contained in~$A ∩ B$, and therefore contained in~$\int(A ∩ B)$.
This shows that
\[
	\int(A) ∩ \int(B) ⊆ \int(A ∩ B) \,.
\]

We have thus shown that~$\int(A ∩ B) = \int(A) ∩ \int(B)$.
It follows that
\begin{align*}
	\closure{A ∪ B}
	&=
	X - (X - \closure{A ∪ B}) \\
	&=
	X - \int(X - (A ∪ B)) \\
	&=
	X - \int((X - A) ∩ (X - B)) \\
	&=
	X - (\int(X - A) ∩ \int(X - B)) \\
	&=
	(X - \int(X - A)) ∪ (X - \int(X - B)) \\
	&=
	\closure{X - (X - A)} ∪ \closure{X - (X - B)} \\
	&=
	\closure{A} ∪ \closure{B} \,.
\end{align*}



\addtocounter{subsubsection}{1}
\subsubsection{}

We have~$A ⊆ B ⊆ \closure{B}$ with~$\closure{B}$ being closed in~$X$, so~$\closure{A} ⊆ \closure{B}$.

We have similarly~$\int(A) ⊆ A ⊆ B$ with~$\int(A)$ being open in~$X$, and thus~$\int(A) ⊆ \int(B)$.



\addtocounter{subsubsection}{-2}
\subsubsection{}

\begin{itemize}

	\item
		We have~$⋂_α A_α ⊆ A_β$ for every index~$β$, therefore~$\int(⋂_α A_α) ⊆ \int(A_β)$ for every index~$β$ by part~(f), and thus~$\int(⋂_α A_α) ⊆ ⋂_β \int(A_β)$.

	\item
		It follows from the previous inclusion~$\int(⋂_α A_α) ⊆ ⋂_α \int(A_α)$ that also
		\[
			\int\biggl( ⋂_α A_α \biggr)
			=
			\int\biggl( \int\biggl( ⋂_α A_α \biggr) \biggr)
			⊆
			\int\biggl( ⋂_α \int(A_α) \biggr)
		\]
		by part~(f).
		For the converse inclusion we observe that~$\int(A_α) ⊆ A_α$ for every index~$α$, therefore~$⋂_α \int(A_α) ⊆ ⋂_α A_α$, and thus, again by part~(f),
		\[
			\int\biggl( ⋂_α \int(A_α) \biggr) ⊆ \int\biggl( ⋂_α A_α \biggr)
		\]

	\item
		We observe that
		\begin{align*}
			{}&
			⋃_α \closure{A_α}
			⊆ \closure{ ⋃_α A_α }
			= \closure{ ⋃_α \closure{A_α} } \\
			\iff{}&
			X - ⋃_α \closure{A_α}
			⊇ X - \closure{ ⋃_α A_α }
			= X - \closure{ ⋃_α \closure{A_α} } \\
			\iff{}&
			⋂_α \int(X - A_α)
			⊇ \int\biggl( ⋂_α {} (X - A_α) \biggr)
			= \int\biggl( ⋂_α \int(X - A_α) \biggr) \,.
		\end{align*}
		We have already proven the third line.

	\item
		We have~$A_α ⊆ ⋃_β A_β$ for every index~$α$, therefore~$\int(A_α) ⊆ \int(⋃_β A_β)$ for every index~$α$ by part~(f), and thus~$⋃_α \int(A_α) ⊆ \int(⋃_β A_β)$.

	\item
		It follows that also
		\begin{align*}
			⋂_α \closure{A_α}
			&=
			⋂_α \closure{X - (X - A_α)} \\
			&=
			X - ⋃_α \int(X - A_α) \\
			&⊇
			X - \int\biggl(⋃_α {} (X - A_α)\biggr) \\
			&=
			\closure{⋂_α {} (X - (X - A_α))}
			=
			\closure{⋂_α A_α}
		\end{align*}

\end{itemize}
Regarding the counterexamples:
\begin{itemize}

	\item
		For every~$x ∈ ℝ$ let~$A_x = \{ x \}$.
		Then
		\[
			⋃_{x ∈ ℚ} \closure{A_x}
			=
			⋃_{x ∈ ℚ} \closure{ \{ x \} }
			=
			⋃_{x ∈ ℚ} {} \{ x \}
			=
			ℚ
			⊊
			ℝ
			=
			\closure{ ℚ }
			=
			\closure{ ⋃_{x ∈ ℚ} A_x } \,.
		\]

	\item
		We get another counterexample by considering~$B_x ≔ ℝ - \{ x \}$, as
		\[
			⋂_{x ∈ ℚ} \int(B_x)
			=
			⋂_{x ∈ ℚ} \int(ℝ - A_x)
			=
			ℝ - ⋃_{x ∈ ℚ} \closure{A_x}
			=
			ℝ - ℚ
		\]
		but
		\[
			\int\biggl( ⋂_{x ∈ ℚ} B_x \biggr)
			=
			\int( ℝ - ℚ )
			=
			∅ \,.
		\]

	\item
		We observe that
		\[
			⋃_{x ∈ ℝ} \int(A_x)
			=
			⋃_{x ∈ ℝ} \int(\{ x \})
			=
			⋃_{x ∈ ℝ} ∅
			=
			∅
			⊊
			ℝ
			=
			\int(ℝ)
			=
			\int\biggl( ⋃_{x ∈ ℝ} A_x \biggr) \,.
		\]

	\item
		Dually,
		\[
			⋂_{x ∈ ℝ} \closure{B_x}
			=
			⋂_{x ∈ ℝ} \closure{ℝ - A_x}
			=
			ℝ - ⋃_{x ∈ ℝ} \int(A_x)
			=
			ℝ - ∅
			=
			ℝ
			⊋
			∅
			=
			\closure{∅}
			=
			\closure{ ⋂_{x ∈ ℝ} B_x } \,.
		\]

\end{itemize}
