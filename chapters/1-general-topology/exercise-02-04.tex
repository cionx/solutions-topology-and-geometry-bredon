\subsection{}

It suffices to show that the collection
\[
	T
	≔
	\{
		U ⊆ Y
	\suchthat
		\text{$f^{-1}(U)$ is open in~$X$}
	\}
\]
is a topology on~$Y$.
This topology does then contain a subbasis for the topology of~$Y$, and does therefore contain all open subsets of~$Y$.
This then means that~$f$ is continuous.

Both~$f^{-1}(Y) = X$ and~$f^{-1}(∅) = ∅$ are open in~$X$, whence both~$Y$ and~$∅$ belong to~$T$.

Let~$U_1$, and~$U_2$ be two sets belonging to~$T$.
This means that the preimages~$f^{-1}(U_1)$ and~$f^{-1}(U_2)$ are both open in~$X$.
It follows that the preimage
\[
	f^{-1}(U_1 ∩ U_2)
	=
	f^{-1}(U_1) ∩ f^{-1}(U_2)
\]
is again open in~$X$.
Hence, the intersection~$U_1 ∩ U_2$ again belongs to~$T$.

Let now~$(U_i)_{i ∈ I}$ be a family of sets belonging to~$T$.
This means that each preimage~$f^{-1}(U_i)$ is open in~$X$.
It follows that the preimage
\[
	f^{-1}\biggl( ⋃_{i ∈ I} U_i \biggr)
	=
	⋃_{i ∈ I} f^{-1}(U_i)
\]
is again open in~$X$.
Hence, the union~$⋃_{i ∈ I} U_i$ again belongs to~$T$.

This shows altogether that~$T$ is indeed a topology on~$Y$, which in turn shows that~$f$ is continuous.
