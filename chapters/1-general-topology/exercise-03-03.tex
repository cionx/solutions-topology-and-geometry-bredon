\subsection{}

Let~$X$ be a metric space.

Suppose first that~$M$ is second countable.
Let~$\basis{B}$ be a countable basis for~$X$.
We may assume that~$∅$ is \emph{not} an element of~$\basis{B}$.
Let~$S$ be a countable subset of~$X$ such that for every~$B ∈ \basis{B}$ there exists some~$x ∈ S$ with~$x ∈ B$. (One may pick a point from each set of~$\basis{B}$.)
The set~$S$ is then dense in~$X$.
Indeed, for every nonempty open subset~$U$ of~$X$ there exists some set~$B ∈ \basis{B}$ with~$B ⊆ U$.
There then exists some point~$x ∈ S$ with~$x ∈ B$, and thus~$x ∈ U$.
This shows that every nonempty open subset of~$X$ intersects the set~$S$, which means precisely that~$S$ is dense in~$X$.

Suppose now conversely that~$X$ admits a countable dense subset~$S$.
We claim that the countable set
\[
	\basis{B}
	≔
	\{
		\ball_ε(x)
	\suchthat
		ε ∈ ℚ,
		ε > 0,
		x ∈ S
	\}
\]
is a basis for~$X$.
Each element of~$\basis{B}$ is an open subset of~$X$, so it remains to show that each open subset~$U$ of~$X$ is a union of elements of~$\basis{B}$.

There exists for every point~$x$ in~$U$ some radius~$ε > 0$ with~$\ball_ε(x) ⊆ U$.
Because~$S$ is dense in~$X$, there then exists some point~$y ∈ S$ that is contained in the open ball~$\ball_{ε / 3}(x)$.
That is, we have~$\dist(x, y) < ε / 3$.
We now have for every rational radius~$δ$ with~$ε/3 < δ < 2 ε / 3$ the containments
\[
	x
	∈
	\ball_{ε / 3}(y)
	⊆
	\ball_δ(y)
	⊆
	\ball_{2 ε / 3}(y)
	⊆
	\ball_ε(x)
	⊆
	U \,,
\]
with~$\ball_δ(y)$ being an element of~$\basis{B}$.
This shows that for each point~$x$ in~$U$ there exists an element~$B_x$ of~$\basis{B}$ with~$x ∈ B_x ⊆ U$.
It follows that~$U = ⋃_{x ∈ U} B_x$.
