\subsection{}

Let~$X_α$ with~$α ∈ A$ be a collection of topological spaces and let~$X ≔ ∏_{α ∈ A} X_α$.



\subsubsection*{Arc connectedness}

Suppose that each~$X_α$ is arcwise connected.

Let~$x$ and~$y$ be two arbitrary points in~$X$.
There exists for every index~$α$ a continuous path~$λ_α \colon [0, 1] \to X_α$ from~$x_α$ to~$y_α$.
The combined map
\[
	λ
	\colon
	[0, 1] \to X \,,
	\quad
	t \mapsto (λ_α(t))_{α ∈ A}
\]
is then a continuous path from~$x$ to~$y$.

This shows that~$X$ is arcwise connected.



\subsubsection*{Connectedness}

Suppose that each~$X_α$ is connected.

Let~$d \colon X \to D$ be a discrete-valued continuous function.
We want to show that the map~$d$ is constant.

For every point~$x$ in~$X$ and every index~$α$ we can consider the line parallel to the~\axis{$α$} running through~$x$;
in other words, the set
\[
	L_{x, α}
	≔
	\{
		y ∈ X
	\suchthat
		\text{$y_β = x_β$ whenever~$β ≠ α$}
	\} \,.
\]
The subspace~$L_{x, α}$ of~$X$ is connected because it is the image of the space~$X_α$ under the continuous map~$λ \colon X_α \to L_{x, α}$ given by
\[
	λ(t)_β
	=
	\begin{cases*}
		x_β & if~$β ≠ α$, \\
		t   & if~$β = α$.\footnotemark
	\end{cases*}
	\footnotetext{
		The map~$λ$ actually restricts to a homeomorphism between~$X_α$ and~$L_{x, α}$, with inverse given by the restriction of the canonical projection from~$X$ to~$X_α$.
	}
\]
The restriction of~$d$ to~$L_{x, α}$ is therefore constant.

This tells us that~$d(x) = d(y)$ whenever two points~$x$ and~$y$ differ in only one coordinate.
It follows with induction that more generally~$d(x) = d(y)$ whenever~$x$ and~$y$ differ by only finitely many coordinates.

Let now~$x$ and~$y$ be two arbitrary points in~$X$ and let~$c ≔ d(x)$.
The set~$U ≔ d^{-1}(c)$ is an open neighbourhood of~$x$ and therefore contains a basis open set~$V$ around~$x$.
This set~$V$ is of the form
\[
	V
	=
	∏_{α ∈ A}
	\begin{cases*}
		V_α & if~$α ∈ B$, \\
		X_α & otherwise,
	\end{cases*}
\]
where~$B$ is some finite subset of~$A$ and each~$V_α$ is an open neighbourhood of~$x_α$ in~$X_α$.
We can now consider the point~$z$ in~$X$ given by the coordinates
\[
	z_α
	=
	\begin{cases*}
		x_α & if~$α ∈ B$, \\
		y_α & otherwise,
	\end{cases*}
\]
The point~$z$ is contained in~$V$ and therefore satisfies~$d(x) = c$ by definition of~$V$.
The points~$z$ and~$y$ differ only in finitely many coordinates, so~$d(z) = d(y)$.
This shows that~$d(x) = c = d(z) = d(y)$.

We have overall shows that~$d$ is constant, which in turn shows that~$X$ is again connected.
