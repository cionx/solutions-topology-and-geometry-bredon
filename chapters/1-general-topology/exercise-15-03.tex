\subsection{}



\subsubsection*{Compact groups}

Let~$B$ be the subgroup of~$G$ generated by~$g$, i.e.,~$B = ⟨g⟩ = \{ g^n \suchthat n ∈ ℤ \}$.
It follows from Proposition~15.10 that the closure of~$B$ is a closed subgroup of~$G$.
It is therefore a compact topological group.

The set~$A$ is closed under multiplication.
It follows as in the proof of Proposition~15.1 that~$\closure{A}$ is again closed under multiplication.

We have~$A ⊆ B$ and therefore~$\closure{A} ⊆ \closure{B}$.
To show that~$\closure{A}$ is a subgroup of~$G$ it suffices to show that~$g^{-1}$ is contained in~$\closure{A}$:
it then follows that~$B ⊆ \closure{A}$ because~$\closure{A}$ is closed under multiplication, therefore~$\closure{B} ⊆ \closure{A}$, and thus~$\closure{A} = \closure{B}$.
(And we already know that~$\closure{B}$ is again a subgroup of~$G$.)

Suppose that~$g^{-1}$ is not contained in~$\closure{A}$.
There then exists a neighbourhood~$U$ of~$g^{-1}$ that is disjoint to~$A$.
The set~$g U$ is a neighbourhood of~$e$ that is disjoint to~$g A$.
It follows that there exists a symmetric open neighbourhood~$V$ of~$e$ with~$V ⊆ g U$.
This neighbourhood is again disjoint to~$g A$, which means that~$g^n ∉ V$ for every~$n ≥ 1$.
As~$V$ is symmetric, we also find that~$g^n ∉ V$ for every~$n ≤ -1$.
The only power of~$g$ contained in~$V$ is therefore~$e$.

It follows that the neutral element~$e$ is isolated in~$\closure{B}$ because
\[
	e
	∈
	V ∩ \closure{B}
	⊆
	\closure{V ∩ B}
	=
	\closure{ \{ e \} }
	=
	\{ e \}
\]
and thus~$V ∩ \closure{B} = \{ e \}$.%
\footnote{
	If~$U$ is an open subset of a topological space~$X$ and~$A$ is an arbitrary subset of~$X$, then~$U ∩ \closure{A} ⊆ \closure{U ∩ A}$.
	Indeed, let~$x$ be an arbitrary point in~$U ∩ \closure{A}$.
	There exists a net~$(x_α)_{α ∈ D}$ in~$A$ with~$x_α \to x$, and this net will eventually be contained in~$U$ because~$U$ is a neighbourhood of~$x$.
	We can therefore consider the truncated subnet~$(x_{α'})_{α' ∈ D'}$ for~$D' ≔ \{ α ∈ D \suchthat x_α ∈ U \}$.
	This is a net in~$U ∩ A$ that again converges to~$x$.
	Hence,~$x ∈ \closure{U ∩ A}$.
}
But~$\closure{B}$ is a topological group, so it further follows that every point of~$\closure{B}$ is isolated.
In other words,~$\closure{B}$ is discrete.
But~$\closure{B}$ is also compact, which now means that~$\closure{B}$ must be finite.
This entails that~$B$ is finite, which means that~$g$ has finite order.
This in turn means that~$g^{-1}$ is contained in~$A$, and therefore also contained in~$\closure{A}$.
But this contradicts the assumption that~$g^{-1} ∉ \closure{A}$.

(There is probably a better formulation of the above argumentation.
But the current version is the best I’ve come up with so far.)



\subsubsection*{Non-compact groups}

The given statement does not necessarily apply to non-compact topological groups.
As an example, we can consider the discrete group~$G ≔ ℤ$ and the element~$g ≔ 1$.
The set~$\{ 0, 1, 2, \dotsc \} = ℕ$ is closed in~$G$, because~$G$ is discrete.
Consequently,~$\closure{ℕ} = ℕ$.
But~$ℕ$ is not a subgroup of~$ℤ$.
