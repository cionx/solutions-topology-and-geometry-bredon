\subsection{}

We denote the given space by~$X$.

All finite subsets of~$X$ are closed (this is part of the definition of the strong order topology), which entails that each singleton is closed.
This tells us that~$X$ is~\Tax{1}.

\begin{definition}
	Let~$X$ be a set.
	A subset~$A$ of~$X$ is \defemph{cofinite} if its complement~$X - A$ is finite.
\end{definition}

By definition of the strong order topology, the space~$X$ has a subbasis given by the subsets
\begin{itemize*}

	\item
		$S_q ≔ \{ p ∈ X \suchthat p < q \}$ with~$q ∈ X$,

	\item
		$T_q ≔ \{ p ∈ X \suchthat p > q \}$ with~$q ∈ X$, and

	\item
		all cofinite subsets of~$X$.

\end{itemize*}
We observe that every subbasis set containing~$x$ or~$y$ is cofinite:
\begin{itemize*}

	\item
		The sets~$S_q$ contains neither~$x$ nor~$y$, because~$x$ and~$y$ are maximal in~$X$.

	\item
		The set~$T_q$ contains~$x$ or~$y$ if and only if~$q ≠ x, y$, i.e., if and only if~$q ∈ ω$.
		And the sets~$T_q$ with~$q ∈ ω$ are finite, because every natural number has only finitely many predecessors.

	\item
		Cofinite subsets of~$X$ are cofinite.

\end{itemize*}
We also observe that finite intersections of cofinite subsets are again cofinite.
It follows that all basis sets (with respect to the above subbasis) that contain either~$x$ or~$y$ are again cofinite.

We note that any two cofinite subsets of~$X$ intersect non-trivially because~$X$ is infinite.
It follows that every two basis sets that contain either~$x$ or~$y$ intersect non-trivially.
It further follows that every two open subsets of~$X$ that contain either~$x$ or~$y$ intersect non-trivially.

This shows that~$X$ is not~\Tax{2}.
