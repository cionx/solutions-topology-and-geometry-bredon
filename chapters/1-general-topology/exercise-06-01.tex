\subsection{}

Let~$(x_n)_n$ be a sequence in a topological space~$X$ that is also universal as a net.

We observe for every point~$x$ in~$X$ that the sequence~$(x_n)_n$ is eventually contained in either~$\{ x \}$ or in~$X - \{ x \}$.
This means that the sequence~$(x_n)_n$ is either eventually constant or assumes every of its values only finitely often.

We need to show that the second case doesn’t occur.
Suppose that it does, i.e., that the sequence~$(x_n)_n$ takes on every values only finitely many times.
By removing repeated values, we arrive at a subsequence~$(y_n)_n$ of~$(x_n)_n$ that takes on each of its values precisely once.
That is,~$y_n ≠ y_m$ whenever~$n ≠ m$.
The subsequence~$(y_n)_n$ is again universal as a net, because subnets of universal nets are universal and because subsequences are subnets.

We can now consider the set~$A ≔ \{ y_n \suchthat \text{$n$ is even} \}$.
The sequence~$(y_n)_n$ alternates between being inside~$A$ and being outside~$A$.
It is therefore neither eventually in~$A$ nor eventually in~$X - A$.
This contradicts~$(y_n)_n$ being universal as a net.
