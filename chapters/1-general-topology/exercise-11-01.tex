\subsection{}

We denote for every topological space~$X$ the set of continuous maps from~$X$ to~$[0, 1]$ by~$F(X)$.



\subsubsection*{Step~1: Construction of~$f^{**} \colon [0, 1]^{F(X)} \to [0, 1]^{F(Y)}$}

\begin{lemma}
	\label{functoriality in exponents}
	Let~$f \colon I \to J$ be a function between two sets~$I$ and~$J$, and let~$X$ be a topological space.
	The induced map
	\[
		f^*
		\colon
		X^J \to X^I \,,
		\quad
		λ \mapsto λ ∘ f \,,
		\quad
		(x_j)_{j ∈ J} \mapsto (x_{f(i)})_{i ∈ I} \,,
	\]
	is continuous with respect to the product topologies on~$X^J$ and~$X^I$.
\end{lemma}

\begin{proof}
	For every index~$i$ let~$π_i \colon X^I \to X$ be the projection onto the~\nth{$i$} coordinate.
	The composite~$π_i ∘ f^* \colon X^J \to X$ is given by projection onto the~\nth{$f(i)$} coordinate.
	This entails that each composite~$π_i ∘ f^*$ is continuous, whence~$f^*$ is continuous.
\end{proof}

We observe that every continuous map~$f \colon X \to Y$ induces a function
\[
	f^*
	\colon
	F(Y) \to F(X) \,,
	\quad
	g \mapsto g ∘ f \,.
\]
According to \cref{functoriality in exponents}, the function~$f^*$ in turn induces a continuous map
\[
	f^{**}
	\colon
	[0, 1]^{F(X)} \to [0, 1]^{F(Y)} \,,
	\quad
	λ \mapsto λ ∘ f^* \,,
\]
given in tuples by
\[
	(x_h)_{h ∈ F(X)} \mapsto (x_{f^*(g)})_{g ∈ F(Y)} = (x_{g ∘ f})_{g ∈ F(Y)} \,.
\]



\subsubsection*{Step~2: commutativity of the induced diagram}

We observe that the map~$f^{**}$ makes the following diagram commute:
\begin{equation}
	\label{diagram for action of stone chech on morphisms}
	\begin{tikzcd}[column sep = large]
		X
		\arrow{r}[above]{f}
		\arrow{d}[left]{Φ_X}
		&
		Y
		\arrow{d}[right]{Φ_Y}
		\\{}
		[0, 1]^{F(X)}
		\arrow{r}[above]{f^{**}}
		&{}
		[0, 1]^{F(Y)}
	\end{tikzcd}
\end{equation}
Indeed, we have for all~$x ∈ X$ and~$g ∈ F(Y)$ the sequence of equalities
\begin{align*}
	(f^{**} ∘ Φ_X)(x)(g)
	&=
	f^{**}( Φ_X(x) )(g) \\
	&=
	(Φ_X(x) ∘ f^*)(g) \\
	&=
	Φ_X(x)( f^*(g) ) \\
	&=
	Φ_X(x)( g ∘ f ) \\
	&=
	(g ∘ f)(x) \\
	&=
	g( f(x) ) \\
	&=
	Φ_Y( f(x) )(g) \\
	&=
	(Φ_Y ∘ f)(x)(g) \,,
\end{align*}
and therefore the equality~$f^{**} ∘ Φ_X = Φ_Y ∘ f$.



\subsubsection*{Step~3: construction of~$β(f)$}

It follows from the continuity of~$f^{**}$ and from the commutativity of the above diagram that
\[
	f^{**}( β(X) )
	=
	f^{**}\Bigl( \closure{Φ_X(X)} \Bigr)
	⊆
	\closure{ f^{**}(Φ_X(X)) }
	=
	\closure{ Φ_Y(f(X)) }
	⊆
	\closure{ Φ_Y(Y) }
	=
	β(Y) \,.%
	\footnote{
		We use here that a map~$f \colon X \to Y$ between topological spaces~$X$ and~$Y$ is continuous if and only if~$f\bigl( \closure{A} \bigr) ⊆ \closure{f(A)}$ for every subset~$A$ of~$X$.
	}
\]
The continuous map
\[
	f^{**} \colon [0, 1]^{F(X)} \to [0, 1]^{F(Y)}
\]
therefore restricts to a map
\[
	β(f) \colon β(X) \to β(Y) \,,
\]
which is still continuous.
The commutative diagram~\eqref{diagram for action of stone chech on morphisms} restricts to the following commutative diagram:
\[
	\begin{tikzcd}
		X
		\arrow{r}[above]{f}
		\arrow{d}[left]{Φ_X}
		&
		Y
		\arrow{d}[right]{Φ_Y}
		\\{}
		β(X)
		\arrow{r}[above]{β(f)}
		&{}
		β(Y)
	\end{tikzcd}
\]



\subsubsection*{Step~4: uniqueness}

Let~$ψ \colon β(X) \to β(Y)$ be another continuous map such that the diagram
\[
	\begin{tikzcd}
		X
		\arrow{r}[above]{f}
		\arrow{d}[left]{Φ_X}
		&
		Y
		\arrow{d}[right]{Φ_Y}
		\\{}
		β(X)
		\arrow{r}[above]{ψ}
		&{}
		β(Y)
	\end{tikzcd}
\]
commutes.
This entails that~$β(f)$ and~$ψ$ agree on the dense subspace~$Φ_X(X)$ of~$β(X)$.
As~$β(Y)$ is a Hausdorff space, it further follows that already~$β(f) ≡ ψ$ on all of~$β(X)$.
Therefore,~$β(f) = ψ$.



\subsubsection*{Step~5: functoriality}

Let~$X$ be a completely regular space.
It follows from the commutativity of the diagram
\[
	\begin{tikzcd}[column sep = large]
		X
		\arrow{r}[above]{\id_X}
		\arrow{d}[left]{Φ_X}
		&
		X
		\arrow{d}[right]{Φ_X}
		\\
		β(X)
		\arrow{r}[above]{\id_{β(X)}}
		&
		β(X)
	\end{tikzcd}
\]
that the map~$\id_{β(X)}$ satisfies the defining property of~$β(\id_X)$ (i.e., being continuous and making the diagram commute).
Therefore,~$β(\id_X) = \id_{β(X)}$.

Let~$X$,~$Y$ and~$Z$ be completely regular spaces and let~$f \colon X \to Y$,~$g \colon Y \to Z$ be continuous map.
We have the following commutative diagram:
\[
	\begin{tikzcd}[column sep = large]
		X
		\arrow[bend left = 35]{rr}[above]{g ∘ f}
		\arrow{r}[above]{f}
		\arrow{d}[left]{Φ_X}
		&
		Y
		\arrow{r}[above]{g}
		\arrow{d}[left]{Φ_Y}
		&
		Z
		\arrow{d}[left]{Φ_Z}
		\\
		β(X)
		\arrow{r}[above]{β(f)}
		\arrow[bend right = 35]{rr}[below]{β(g) ∘ β(f)}
		&
		β(Y)
		\arrow{r}[above]{β(g)}
		&
		β(Z)
	\end{tikzcd}
\]
The commutativity of the outer square diagram
\[
	\begin{tikzcd}[column sep = huge]
		X
		\arrow{r}[above]{g ∘ f}
		\arrow{d}[left]{Φ_X}
		&
		Z
		\arrow{d}[right]{Φ_Z}
		\\
		β(X)
		\arrow{r}[above]{β(g) ∘ β(f)}
		&
		β(Z)
	\end{tikzcd}
\]
tells us that the composite~$β(g) ∘ β(f)$ satisfies the defining property of the induced map~$β(g ∘ f)$, whence~$β(g ∘ f) = β(g) ∘ β(f)$.
