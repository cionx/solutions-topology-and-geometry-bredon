\subsection{}



\subsubsection{}

\paragraph{Well-definedness}
Let~$\class{ (x_n)_n }$ and~$\class{ (y_n)_n }$ be two elements of~$Y$.
Both~$(x_n)_n$ and~$(y_n)_n$ are Cauchy sequences, so there exist some index~$N$ with
\[
	\dist(x_n, x_m), \dist(y_n, y_m) < \frac{ε}{2}
	\qquad
	\text{for all~$n, m ≥ N$} \,.
\]
It follows from the triangle inequality and reverse triangle inequality that
\begin{align*}
	{}&
	\abs{\dist(x_n, y_n) - \dist(x_m, y_m)}
	\\
	≤{}&
	\abs{\dist(x_n, y_n) - \dist(x_m, y_n)}
	+ \abs{\dist(x_m, y_n) - \dist(x_m, y_m)}
	\\
	≤{}&
	\dist(x_n, x_m) + \dist(y_n, y_m)
	\\
	<{}&
	\frac{ε}{2} + \frac{ε}{2}
	\\
	={}&
	ε
\end{align*}
for all~$n, m ≥ N$.
This shows that the real-valued sequence~$( \dist(x_n, y_n) )_n$ is again a Cauchy sequence and thus converges.
The limit~$\lim_{n \to ∞} \dist(x_n, y_n)$ is therefore well-defined.

\begin{remark}
	Alternatively, we can endow~$X × X$ with the metric~$\dist_1$ given by~$\dist_1( (x_1, x_2), (x'_1, x'_2)) ≔ \dist(x_1, x'_1) + \dist(x_2, x_2)$.
	The sequence~$((x_n, y_n))_n$ in~$X × X$ is then again a Cauchy sequence.
	It follows from the upcoming \cref{metric is lipschitz continuous} that the distance function~$\dist \colon X × X \to ℝ$ is uniformly continuous, whence it follows from \cref{effect of uniformly continuous maps on sequences} that the sequence~$(\dist(x_n, y_n))_n$ is again a Cauchy sequence.
\end{remark}

\paragraph{Positivity}
Let~$\class{ (x_n)_n }$ and~$\class{ (y_n)_n }$ be two elements of~$Y$.
We have for every index~$n$ the inequality~$\dist(x_n, y_n) ≥ 0$ and therefore the resulting inequality~$\lim_{n \to ∞} \dist(x_n, y_n) ≥ 0$.
This means that~$\dist(\class{(x_n)_n}, \class{(y_n)_n}) ≥ 0$.

We also have the sequence of equivalences
\begin{align*}
	{}&
	\dist(\class{(x_n)_n}, \class{(y_n)_n}) = 0
	\\
	\iff{}&
	\textstyle \lim_{n \to ∞} \dist(x_n, y_n) = 0
	\\
	\iff{}&
	\text{$(x_n)_n$ and $(y_n)_n$ are equivalent}
	\\
	\iff{}&
	\class{ (x_n)_n } = \class{ (y_n)_n } \,.
\end{align*}

\paragraph{Symmetry}
We have for any two elements~$\class{ (x_n)_n }$ and~$\class{ (y_n)_n }$ of~$Y$ the sequence of equalities
\begin{align*}
	\dist( \class{(x_n)_n}, \class{(y_n)_n} )
	={}&
	\textstyle \lim_{n \to ∞} \dist(x_n, y_n)
	\\
	={}&
	\textstyle \lim_{n \to ∞} \dist(y_n, x_n)
	\\
	={}&
	\dist( \class{(y_n)_n}, \class{(x_n)_n} ) \,.
\end{align*}

\paragraph{Triangle inequality}
We have for all elements~$\class{ (x_n)_n }$,~$\class{ (y_n)_n }$ and~$\class{ (z_n)_n }$ of~$Y$ the sequence of (in)equalities
\begin{align*}
	{}&
	\dist( \class{(x_n)_n}, \class{(y_n)_n} ) + \dist( \class{(y_n)_n}, \class{(z_n)_n} )
	\\
	={}&
	\textstyle \lim_{n \to ∞} \dist(x_n, y_n) + \lim_{n \to ∞} \dist(y_n, z_n)
	\\
	={}&
	\textstyle \lim_{n \to ∞} {} \bigl( \dist(x_n, y_n) + \dist(y_n, z_n) \bigr)
	\\
	≥{}&
	\textstyle \lim_{n \to ∞} \dist(x_n, z_n)
	\\
	={}&
	\dist( \class{(x_n)_n}, \class{(z_n)_n} ) \,.
\end{align*}



\subsubsection{}

We have for any two points~$x$ and~$y$ in~$X$ the equalities
\[
	\dist( f(x), f(y) )
	=
	\dist( \class{(x)_n}, \class{(y)_n}
	=
	\lim_{n \to ∞} \dist(x, y)
	=
	\dist(x, y) \,.
\]
This shows that the map~$f$ is an isometry.

That~$f(X)$ is dense in~$Y$ follows from the following \lcnamecref{cauchy sequence converges to itself}.

\begin{claim}
	\label{cauchy sequence converges to itself}
	Let~$y$ be an element of~$Y$ with~$y = \class{ (x_n)_n }$.
	Then~$f(x_m) \to y$ as~$m \to ∞$.
\end{claim}

\begin{proof}
	Let~$ε > 0$.
	There exists an index~$N$ with~$\dist(x_n, x_m) < ε$ for all~$n, m ≥ N$ because~$(x_n)_n$ is a Cauchy sequence.
	It follows that
	\begin{align*}
		\dist( y, f(x_m) )
		&=
		\dist( \class{(x_n)_n}, f(x_m) ) \\
		&=
		\textstyle \lim_{n \to ∞} \dist(x_n, x_m) \\
		&=
		\textstyle \limsup_{n \to ∞} \dist(x_n, x_m) \\
		&≤
		ε
	\end{align*}
	for every~$m ≥ N$.
\end{proof}



\subsubsection{}

Let~$(y_n)_n$ be a Cauchy sequence in~$Y$.
We need to show that the sequence~$(y_n)_n$ converges in~$Y$.

The sequence~$(y_n)_n$ is a sequence of (equivalence classes of) sequences.
We will essentially use a diagonal argument to construct another Cauchy sequence in~$X$, and thus a point in~$Y$, towards which the sequence~$(y_n)_n$ is going to converge.

A Cauchy sequence converges if and only if it has a convergent subsequence, so we may pass from the entire sequence~$(y_n)_n$ to any one of its subsequences.
This allows us to additionally assume that
\begin{equation}
	\label{inequality for subsequence}
	\dist( y_{n_1}, y_{n_2} ) < \frac{1}{n}
	\qquad
	\text{for all~$n_1, n_2 ≥ n ≥ 1$.}
\end{equation}

Each term~$y_n$ is of the form~$\class{(x_{n, k})_k}$ for a Cauchy sequence~$(x_{n, k})_k$ in~$X$.
We have already seen in \cref{cauchy sequence converges to itself} that
\begin{equation}
	\label{convergence from previous exercise}
	y_n = \lim_{k \to ∞} f(x_{n, k})
	\qquad
	\text{for every~$n$} \,.
\end{equation}
This means that there exists for every index~$n$ some index~$K_n ≥ 1$ with
\begin{equation}
	\label{inequality from convergence}
	\dist( f(x_{n, k}), y_n ) < \frac{1}{n}
	\qquad
	\text{for all~$k ≥ K_n$} \,.
\end{equation}

For every~$n ≥ 1$ let
\[
	z_n ≔ x_{n, K_n} \,,
\]
so that~$(z_n)_n$ is a sequence in~$X$.
We claim that~$(z_n)_n$ is a Cauchy sequence, and that the sequence~$(y_n)_n$ in~$Y$ converges to~$\class{(z_n)_n}$.

\paragraph{Cauchy sequence}
We have for all~$n_1, n_2 ≥ n ≥ 1$ the (in)equalities
\begin{align*}
	{}&
	\dist(z_{n_1}, z_{n_2})
	\\
	={}&
	\dist\bigl( f(z_{n_1}), f(z_{n_2}) \bigr)
	\\
	={}&
	\dist\bigl( f(x_{n_1, K_{n_1}}), f(x_{n_2, K_{n_2}}) \bigr)
	\\
	≤{}&
		\underbrace{
			\dist\bigl( f(x_{n_1, K_{n_1}}), y_{n_1} \bigr)
		}_{\text{$≤ 1 / n_1$ by~\eqref{inequality from convergence}}}
	+ \underbrace{
			\dist\bigl( y_{n_1}, y_{n_2} \bigr)
		}_{\text{$≤ 1 / n$ by~\eqref{inequality for subsequence}}}
	+ \underbrace{
			\dist\bigl( y_{n_2}, f(x_{n_2, K_{n_2}}) \bigr)
		}_{\text{$≤ 1 / n_2$ by~\eqref{inequality from convergence}}}
	\\
	≤{}&
	\frac{1}{n_1} + \frac{1}{n} + \frac{1}{n_2}
	\\[0.4em]
	≤{}&
	\frac{3}{n} \,,
\end{align*}
which shows that the sequence~$(z_n)_n$ is a Cauchy sequence.

\paragraph{Convergence}
We can therefore consider the element~$z ≔ \class{(z_n)_n}$ of~$Y$.
We have for every~$n ≥ 1$ the (in)equalities
\begin{align*}
	{}&
	\dist(y_n, z)
	\\
	={}&
	\lim_{m \to ∞} \dist(x_{n, m}, z_m)
	\\
	={}&
	\lim_{m \to ∞} \dist(x_{n, m}, x_{m, K_m})
	\\
	={}&
	\lim_{m \to ∞} \dist( f(x_{n, m}), f(x_{m, K_m}) )
	\\
	={}&
	\limsup_{m \to ∞} \dist( f(x_{n, m}), f(x_{m, K_m}) )
	\\
	≤{}&
	\limsup_{m \to ∞} {}
	\Bigl(
		\underbrace{
			\dist\bigl( f(x_{n, m}), y_n \bigr)
		}_{
			\text{$\to 0$ by~\eqref{convergence from previous exercise}}
		}
		+ \underbrace{
			\dist\bigl( y_n, y_m \bigr)
		}_{
			\text{$≤ 1 / n$ for large~$m ≥ n$ by~\eqref{inequality for subsequence}}
		}
		+ \underbrace{
			\dist\bigl( y_m, f(x_{m, K_m}) \bigr)
		}_{
			\text{$≤ 1 / m$ by~\eqref{inequality from convergence}}
		}
	\Bigr)
	\\
	≤{}&
	\frac{1}{n} \,.
\end{align*}
This shows that~$y_n \to z$ as~$n \to ∞$.



\subsubsection{}

\begin{proposition}
	\label{uniformly continuous maps can be extended}
	Let~$Y$ be a metric space and let~$Z$ be a complete metric space.
	Let~$X$ be a dense subspace of~$Y$ and let~$g \colon X \to Z$ be a uniformly continuous map.
	There exists a unique continuous extension of~$g$ to a map from~$Y$ to~$Z$, and~$g$ is again uniformly continuous.
\end{proposition}

The proof of the above \lcnamecref{uniformly continuous maps can be extended} will depend on the following auxiliary results about the effect of uniformly continuous maps on sequences:

\begin{lemma}
	\label{effect of uniformly continuous maps on sequences}
	Let~$X$ and~$Z$ be a metric spaces and let~$g \colon X \to Z$ be a uniformly continuous map.
	\begin{enumerate}

		\item
			The map~$g$ maps Cauchy sequences in~$X$ to Cauchy sequences in~$Z$.

		\item
			Let~$(x_n)_n$ and~$(y_n)_n$ be two sequences in~$X$ with~$\dist(x_n, y_n) \to 0$ as~$n \to ∞$.
			Then also~$\dist( g(x_n), g(y_n) ) \to 0$ as~$n \to ∞$.

	\end{enumerate}
\end{lemma}

\begin{proof}
	\leavevmode
	\begin{enumerate}

		\item
			Let~$(x_n)_n$ be a Cauchy sequence in~$X$.
			We need to show that the image sequence~$(g(x_n))_n$ in~$Z$ is again a Cauchy sequence.

			Let~$ε > 0$.
			There exists some~$δ > 0$ such that~$\dist( g(x), g(y) ) < ε$ for all~$x, y ∈ X$ with~$\dist(x, y) < δ$.
			There then exists some index~$N$ with~$\dist(x_n, x_m) < δ$ whenever~$n, m ≥ N$.
			Therefore,~$\dist(g(x_n), g(x_m)) < ε$ whenever~$n, m ≥ N$.
			This shows that the sequence~$( g(x_n) )_n$ is again a Cauchy sequence.

		\item
			Let again~$ε > 0$.
			There exists some~$δ > 0$ such that~$\dist(g(x), g(y)) < ε$ for all~$x, y ∈ X$ with~$\dist(x, y) < δ$.
			There then exists some index~$N$ with~$\dist(x_n, y_n) < δ$ whenever~$n ≥ N$.
			Therefore,~$\dist(g(x_n), g(y_n)) < ε$ whenever~$n ≥ N$.
			
			This shows that~$\limsup_{n \to ∞} \dist(g(x_n), g(y_n)) ≤ ε$ for every~$ε > 0$, and consequently~$\limsup_{n \to ∞} \dist(g(x_n), g(y_n)) = 0$.
		\qedhere

	\end{enumerate}
\end{proof}

\begin{proof}[Proof of \cref{uniformly continuous maps can be extended}]
	We first explain how to construct the function~$f$, and also show that~$f$ is well-defined.
	We then show that~$f$ is indeed an extension of~$g$, and that~$f$ is again uniformly continuous.
	Finally, we show the uniqueness of~$f$.

	\paragraph{Construction}
	Let~$y$ be an arbitrary point in~$Y$.
	There exists a sequence~$(x_n)_n$ in~$X$ that converges to~$y$ because~$X$ is dense in~$Y$.
	It follows from \cref{effect of uniformly continuous maps on sequences} that the resulting sequence~$(g(x_n))_n$ in~$Z$ is again a Cauchy sequence.
	It therefore converges because~$Z$ is complete.
	Let~$f(x) ≔ \lim_{n \to ∞} g(x_n)$.

	\paragraph{Well-definedness}
	Let~$(x_n)_n$ and~$(x'_n)$ be two sequences in~$X$ that converge to the same point~$y$ in~$Y$.
	This entails that~$\dist(x_n, x'_n) \to 0$ as~$n \to ∞$.
	It follows from \cref{effect of uniformly continuous maps on sequences} that also~$\dist(g(x_n), g(x'_n)) \to 0$ as~$n \to ∞$.
	This in turn implies that the limits of the two sequences~$(g(x_n))_n$ and~$(g(x'_n))_n$ are equal.

	\paragraph{Extension}
	Let~$x$ be an arbitrary point in~$X$.
	The constant sequence~$(x)_n$ converges to~$x$, whence~$f(x) = \lim_{n \to ∞} g(x) = g(x)$.

	\paragraph{Uniform continuity}
	Let~$ε > 0$.
	There exists some distance~$δ > 0$ such that~$\dist(g(x), g(x')) < ε$ whenever~$\dist(x, x') < δ$.
	
	Let~$y$ and~$y'$ be two arbitrary points in~$Y$ with~$\dist(y, y') < δ / 3$.
	There exist sequences~$(x_n)_n$ and~$(x'_n)_n$ in~$X$ that convert to~$y$ and~$y'$ respectively.
	There exists some index~$N$ with~$\dist(x_n, y) < δ / 3$ and~$\dist(x'_n, y') < δ / 3$ for all~$n ≥ N$.
	It follows from the triangle inequality that~$\dist(x_n, x'_n) < δ$ for all~$n ≥ N$, and therefore~$\dist(g(x_n), g(x'_n)) < ε$ for all~$n ≥ N$.
	We observe that therefore
	\begin{align*}
		{}&
		\dist( f(y), f(y') )
		\\
		={}&
		\limsup_{n \to ∞} \dist( f(y), f(y') )
		\\
		≤{}&
		\limsup_{n \to ∞} {}
		(
			\underbrace{ \dist( f(y), g(x_n) ) }_{\to 0}
			+ \underbrace{ \dist( g(x_n), g(x'_n) ) }_{
				\text{$< ε$ for large~$n$}
			}
			+ \underbrace{ \dist( g(x'_n), f(y') ) }_{\to 0}
		)
		\\
		≤{}& ε \,.
	\end{align*}

	We have thus shown that~$\dist( f(y), f(y') ) ≤ ε$ whenever~$\dist(y, y') < δ / 3$.
	This shows that the function~$f$ is uniformly continuous.

	\paragraph{Uniqueness}
	Let~$f'$ be another extension of~$g$ to a continuous map from~$Y$ to~$Z$.
	The set~$C ≔ \{ y ∈ Y \suchthat f(y) = f'(y) \}$ is closed in~$Y$ because~$Z$ is a Hausdorff space and both~$f$ and~$f'$ are continuous.
	The dense subspace~$X$ of~$Y$ is contained in~$C$, whence it follows that~$C = Y$.
	This means that~$f = f'$.
\end{proof}

We will also use the following observation:

\begin{lemma}
	\label{metric is lipschitz continuous}
	Let~$X$ be a metric space.
	The metric function~$\dist \colon X × X \to ℝ$ is (Lipschitz) continuous.
\end{lemma}

\begin{proof}
	The product topology on~$X × X$ is induced by the metric~$\dist_1$ given by
	\[
		\dist_1( (x_1, x_2),  (x'_1, x'_2) )
		=
		\dist(x_1, x'_1) + \dist(x_2, x'_2) \,.
	\]
	It follows for all~$(x_1, x_2), (x'_1, x'_2) ∈ X × X$ from the (reverse) triangle inequality that
	\begin{align*}
		\SwapAboveDisplaySkip
		{}&
		\abs{ \dist(x_1, x_2) - \dist(x'_1, x'_2) }
		\\
		≤{}&
		\abs{ \dist(x_1, x_2) - \dist(x'_1, x_2) }
		+ \abs{ \dist(x'_1, x_2) - \dist(x'_1, x'_2) }
		\\
		≤{}&
		\dist(x_1, x'_1) + \dist(x_2, x'_2)
		\\
		={}&
		\dist_1( (x_1, x_2), (x'_1, x'_2) ) \,.
	\end{align*}
	This shows that with respect to the metric~$\dist_1$, the metric of~$X$ is Lipschitz continuous.
\end{proof}

The isometry~$g$ is uniformly continuous.
It therefore follows from \cref{uniformly continuous maps can be extended} that there exists a unique continuous extension~$h \colon Y \to Z$ of~$g$ along~$f$, i.e., a unique factorization~$g = h ∘ f$ with~$h$ continuous:
\[
	\begin{tikzcd}
		Y
		\arrow[dashed]{r}[above]{h}
		&
		Z
		\\
		X
		\arrow{ur}[below right]{g}
		\arrow{u}[left]{f}
		&
		{}
	\end{tikzcd}
\]

It remains to show that~$h$ is again an isometry.
We present two similar argumentations:
\begin{itemize}

	\item
		Let~$y$ and~$y'$ be two points in~$Y$.
		There exist sequences~$(x_n)_n$ and~$(x'_n)_n$ in~$X$ with~$y = \class{(x_n)_n}$ and~$y' = \class{(x'_n)_n}$.
		We have
		\[
			\dist(y, y')
			=
			\lim_{n \to ∞} \dist(x_n, x'_n)
		\]
		by definition of the metric on~$Y$.
		We also know from \cref{cauchy sequence converges to itself} that~$f(x_n) \to y$ and~$f(x'_n) \to y'$ as~$n \to ∞$.
		It follows that~$g(x_n) = h(f(x_n)) \to h(y)$ for~$n \to ∞$ and similarly~$g(x'_n) \to h(y')$.
		It further follows that
		\[
			\dist(h(y), h(y'))
			=
			\lim_{n \to ∞} \dist(g(x_n), g(x'_n))
			=
			\lim_{n \to ∞} \dist(x_n, x'_n)
		\]
		by \cref{metric is lipschitz continuous} and because~$g$ is an isometry.
		Therefore,
		\[
			\dist(y, y')
			=
			\lim_{n \to ∞} \dist(x_n, x'_n)
			=
			\dist(h(y), h(y')) \,.
		\]
		This shows that~$h$ is an isometry.

	\item
		Let~$y$ and~$y'$ be two points in~$f(X)$.
		This means that there exists points~$x$ and~$x'$ in~$X$ with~$y = f(x)$ and~$y' = f(x')$.
		We then have
		\begin{align*}
			\dist(h(y), h(y'))
			&=
			\dist(h(f(x)), h(f(x'))) \\
			&=
			\dist(g(x), g(x')) \\
			&=
			\dist(x, x') \\
			&=
			\dist(f(x), f(x')) \\
			&=
			\dist(y, y')
		\end{align*}
		because both~$g$ and~$f$ are isometries.
		This shows that the restriction of~$h$ to~$f(X)$ is an isometry.
		It follows from the following \lcnamecref{isometry can be checked on a dense subspace} that~$h$ is an isometry.

		\begin{lemma}
			\label{isometry can be checked on a dense subspace}
			Let~$f \colon X \to Y$ be a continuous map between metric spaces~$X$ and~$Y$.
			Let~$D$ be a dense subspace of~$X$ such that the restriction~$\restrict{f}{D}$ is an isometry.
			Then~$f$ is an isometry.
		\end{lemma}

		\begin{proof}
			Let~$x$ and~$x'$ be two points in~$X$.
			There exist sequences~$(x_n)_n$ and~$(x'_n)_n$ in~$D$ with~$x_n \to x$ and~$x'_n \to x'$ as~$n \to ∞$.
			It follows from \cref{metric is lipschitz continuous} that
			\[
				\dist(x, x')
				=
				\lim_{n \to ∞} \dist(x_n, x'_n) \,.
			\]
			It also follows from the continuity of~$f$ that~$f(x_n) \to f(x)$ and~$f(x'_n) \to f(x')$ as~$n \to ∞$.
			Therefore,
			\[
				\dist(f(x), f(x'))
				=
				\lim_{n \to ∞} \dist(f(x_n), f(x'_n))
				=
				\lim_{n \to ∞} \dist(x_n, x'_n)
			\]
			by \cref{metric is lipschitz continuous} and because~$f$ is an isometry.
			It follows that
			\[
				\dist(f(x), f(x'))
				=
				\lim_{n \to ∞} \dist(f(x_n), f(x'_n))
				=
				\dist(x, x') \,,
			\]
			which shows that~$f$ is an isometry.
		\end{proof}
\end{itemize}



\subsubsection{}

Let~$z$ be an arbitrary point in~$Z$.
There exists by assumption a sequence~$(z_n)_n$ in~$g(X)$ with~$z_n \to z$ as~$n \to ∞$.
The sequence~$(z_n)_n$ is in particular a Cauchy sequence.

There exists for every index~$n$ a point~$x_n$ in~$X$ with~$z_n = g(x_n)$.
The map~$g$ is an isometry, whence the sequence~$(x_n)_n$ in~$X$ is again a Cauchy sequence.

For every index~$n$ let~$y_n ≔ f(x_n)$.
The sequence~$(y_n)_n$ in~$Y$ is again a Cauchy sequence because~$f$ is an isometry.
This sequence converges towards some point~$y$ because~$Y$ is complete.
(This point~$y$ is more explicitly given by~$\class{(x_n)_n}$.)

It follows from the continuity of~$h$ that the sequence~$(h(y_n))_n$ converges to~$h(y)$.
But we have
\[
	h(y_n)
	=
	h(f(x_n))
	=
	g(x_n)
	=
	z_n
\]
for every index~$n$, whence the sequence~$(h(y_n))_n$ is the sequence~$(z_n)_n$.
We now have~$h(y) = z$ by the uniqueness of limits in metric spaces.
