\subsection{}

Suppose such a topology were to exist on~$X$.
It then follows for every point~$x$ in~$X$ and every subset~$N$ of~$X$ that~$N$ is a neighbourhood of~$x$ if and only if there exists some~$N' ∈ \basis{N}(x)$ with~$N' ⊆ N$.
This shows that the notion of \enquote{neighbourhood} for the described topology is uniquely determined by~$\basis{N}$.
A set is open if and only if it is a neighbourhood for each of its points, so it further follows that the open subsets are uniquely determined by~$\basis{N}$.
This shows the uniqueness of the described topology.

To show the existence of the described topology, we now define a subset~$U$ of~$X$ to be open if and only if
\begin{equation}
	\label{condition for openness}
	\text{there exists for every~$x ∈ U$ some~$N ∈ \basis{N}(x)$ with~$N ⊆ U$.}
\end{equation}
We claim that this notion on openness defines a topology on~$X$, such that for every point~$x$ in~$X$ the collection~$\basis{N}(x)$ is a neighbourhood basis of~$x$.

The empty set is vacuously open.
The entire space~$X$ satisfies condition~\eqref{condition for openness} by properties~(1) and~(3) of~$\basis{N}$.

Let~$U$ and~$V$ be two open subsets of~$X$.
For every point~$x$ in the intersection~$U ∩ V$ there exist sets~$N ∈ \basis{N}(x)$ and~$M ∈ \basis{N}(x)$ with~$N ⊆ U$ and~$M ⊆ V$.
By property~(2) there exists some~$P ∈ \basis{N}(x)$ with both~$P ⊆ N$ and~$P ⊆ M$.
It follows that both~$P ⊆ U$ and~$P ⊆ V$, and thus~$P ⊆ U ∩ V$.
This shows that for every~$x ∈ U ∩ V$ there exists some~$P ∈ \basis{N}(x)$ with~$P ⊆ U ∩ V$,
which means that~$U ∩ V$ is again open in~$X$.

Let now~$(U_i)_{i ∈ I}$ be a family of open subsets of~$X$.
For every point~$x$ of the union~$⋃_{i ∈ I} U_i$ there exists some index~$j$ with~$x ∈ U_j$.
There then exists some~$N ∈ \basis{N}(x)$ with~$N ⊆ U_j$, and thus~$N ⊆ ⋃_{i ∈ I} U_i$.
This shows that for every~$x ∈ ⋃_{i ∈ I} U_i$ there exists some~$N ∈ \basis{N}(x)$ with~$N ⊆ ⋃_{i ∈ I} U_i$, which means that~$⋃_{i ∈ I} U_i$ is again open in~$X$.

We have altogether shown that we have defined a topology on~$X$.

It remains to show that for every point~$x$ in~$X$, the collection~$\basis{N}(x)$ is a neighbourhood basis of~$x$ with respect to the defined topology.
We hence need to show that
\begin{itemize*}

	\item
		every set belonging to~$\basis{N}(x)$ is a neighbourhood of~$x$, and that

	\item
		each neighbourhood of~$x$ contains a set belonging to~$\basis{N}(x)$.

\end{itemize*}

For the second point, let~$N$ be a neighbourhood of~$x$.
This means that there exists an open subset~$U$ of~$X$ with~$x ∈ U ⊆ N$.
That~$U$ is open and contains~$x$ means that there exists an element~$N'$ of~$\basis{N}(x)$ with~$N' ⊆ U$.
It then follows that~$N' ⊆ N$.

It doesn’t seem to be possible to prove the first point from just the given three conditions.
Indeed, the axioms of a neighbourhood basis typically entail another, fourth axiom:
\begin{enumerate*}[label = (\arabic*), start = 4]

	\item
		There exists for every~$N ∈ \basis{N}(x)$ some~$M ∈ \basis{N}(x)$ such that
		\begin{itemize}

			\item
				$M ⊆ N$ and

			\item
				for every~$y ∈ M$ there exists some~$P ∈ \basis{N}(y)$ with~$P ⊆ M$.

		\end{itemize}
\end{enumerate*}
In our context, this fourth axiom means precisely that~$\basis{N}(x)$ consists of neighbourhoods of~$x$.
