\subsection{}



\subsubsection*{Proposition~15.8}

The map
\[
	f
	\colon
	G × G \to G \,,
	\quad
	(h_1, h_2) \mapsto h_1 g h_2
\]
is continuous with~$f( (e, e) ) = g$.
It follows that there exists a neighbourhood~$V'''$ of~$(e, e)$ with~$f(V''') ⊆ U$.
There exist neighbourhoods~$V''_1$ and~$V''_2$ of the identity of~$G$ with~$V''_1 × V''_2 ⊆ V'''$.
The intersection~$V' ≔ V''_1 ∩ V''_2$ is again a neighbourhood of the identity, and thus contains a symmetric neighbourhood~$V$ of the identity.
It holds altogether that
\[
	V g V^{-1}
	=
	V g V
	⊆
	V' g V'
	⊆
	V''_1 g V''_2
	=
	f(V''_1 × V''_2)
	⊆
	f(V''')
	⊆
	U \,.
\]



\subsubsection*{Proposition~15.9}

The map
\[
	f
	\colon
	\underbrace{ G × \dotsb × G }_n \to G \,,
	\quad
	(g_1, \dotsc, g_n) \mapsto g_1 \dotsm g_n
\]
is continuous with~$f((e, \dotsc, e)) = e$.
It follows that there exists a neighbourhood~$V'''$ of~$(e, \dotsc, e)$ with~$f(V''') ⊆ U$.
There exist neighbourhoods~$V''_1, \dotsc, V''_n$ of the identity of~$G$ with~$V''_1 × \dotsb × V''_n ⊆ V'''$.
The intersection~$V' ≔ V''_1 ∩ \dotsb ∩ V''_n$ is again a neighbourhood of the identity, whence there exists a symmetric neighbourhood~$V$ of the identity contained in~$V'$.
It holds altogether that
\[
	V^n
	⊆
	(V')^n
	=
	f(V' × \dotsb × V')
	⊆
	f(V''_1 × \dotsb × V''_n)
	⊆
	f(V''')
	⊆
	U \,.
\]
