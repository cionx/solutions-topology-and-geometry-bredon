\subsection{}

The given topology is finer than the euclidean topology, which is~\Tax{2} by the upcoming Exercise~9.
Thus,~$X$ is also~\Tax{2}.

Let~$x_0 ≔ (0, 0)$, and for every~$a > 0$ let~$D_a$ be the subbasis open set
\[
	D_a ≔ \{ (x, y) ∈ ℝ^2 \suchthat x^2 + y^2 < a, y ≠ 0 \} ∪ \{ x \} \,.
\]
We make the following two observations:

\begin{claim}
	\label{contains intersection of ball and disk}
	Let~$x$ be a point in~$X$ and let~$U$ be a neighbourhood of~$X$.
	There exist radii~$ε > 0$ and~$a > 0$ with either
	\[
		x ∈ \ball_ε(x) ⊆ U
		\quad\text{or}\quad
		x ∈ \ball_ε(x) ∩ D_a ⊆ U \,.
	\]
\end{claim}

\begin{proof}
	We may assume that~$U$ is a basis open set.
	This then means that~$U$ is a finite intersection of subbasis open sets:
	we have
	\[
		U = V_1 ∩ \dotsb ∩ V_n ∩ D_{a_1} ∩ \dotsb ∩ D_{a_m}
	\]
	for some standard open subsets~$V_1, \dotsc, V_n$ of~$ℝ^2$ and some radii~$a_1, \dotsc, a_m > 0$.
	Each standard open set~$V_i$ must contain the point~$x$ because~$U$ contains~$x$.
	There hence exist radii~$ε_1, \dotsc, ε_n > 0$ with~$\ball_{ε_i}(x) ⊆ V_i$ for every index~$i$.
	It follows that
	\begin{equation}
		\label{intersection of balls and disks}
		x ∈ \ball_{ε_1}(x) ∩ \dotsb ∩ \ball_{ε_n}(x) ∩ D_{a_1} ∩ \dotsb ∩ D_{a_m} ⊆ U \,.
	\end{equation}
	We now distinguish between the following four cases:
	\begin{itemize}
	
		\item
			If both~$n = 0$ and~$m = 0$, then the set in~\eqref{intersection of balls and disks} is the entire space~$X$, whence also~$U = X$.
			We then have~$x ∈ \ball_ε(x) ⊆ U$ for every radius~$ε > 0$.

		\item
			If both~$n = 0$ and~$m ≠ 0$, then the middle term in~\eqref{intersection of balls and disks} equals~$D_a$ for the radius~$a = \min(a_1, \dotsc, a_m)$.
			We then have~$x ∈ \ball_ε(x) ∩ D_a ⊆ U$ for every radius~$ε > 0$.

		\item
			If both~$n ≠ 0$ and~$m = 0$, then the middle term in~\eqref{intersection of balls and disks} equals~$\ball_ε(x)$ for the radius~$ε = \min(ε_1, \dotsc, ε_n)$.

		\item
			If both~$n ≠ 0$ and~$m ≠ 0$, then the middle term in~\eqref{intersection of balls and disks} equals~$\ball_ε(x) ∩ D_a$ for the radii~$ε = \min(ε_1, \dotsc, ε_n)$ and~$a = \min(a_1, \dotsc, a_m)$.
		\qedhere

	\end{itemize}
\end{proof}

\begin{claim}
	\label{contains disk neighbourhood}
	Let~$V$ be a neighbourhood of~$x_0$.
	There exists some radius~$a > 0$ with~$x_0 ∈ D_a ⊆ V$.
\end{claim}

\begin{proof}
	We know from \cref{contains intersection of ball and disk} that either
	\begin{itemize*}

		\item
			there exists a radius~$ε > 0$ with~$x_0 ∈ \ball_ε(x_0) ⊆ V$, or

		\item
			there exist radii~$ε > 0$ and~$a > 0$ with~$x_0 ∈ \ball_ε(x_0) ∩ D_a ⊆ V$.

	\end{itemize*}
	We have~$x_0 ∈ D_a$ for every radius~$a > 0$, so in the first case we moreover have~$x_0 ∈ \ball_ε(x_0) ∩ D_a ⊆ V$ for every radius~$a > 0$.
	We find in either case that
	\[
		x_0 ∈ \ball_ε(x_0) ∩ D_a ⊆ V
	\]
	for some radii~$ε > 0$ and~$a > 0$.
	But the intersection~$\ball_ε(x_0) ∩ D_a$ is~$D_{\min(ε^2, a)}$ because both~$\ball_ε(x_0)$ and~$D_a$ are centered around~$x_0$.
\end{proof}

We claim that the point~$x_0$ and the closed subset~$C ≔ X - D_1$ of~$X$, which doesn’t contain~$x_0$, cannot be separated by disjoint open subsets of~$X$.
To this end, let~$U$ and~$V$ be two open subsets of~$X$ with~$C ⊆ U$ and~$x ∈ V$.
We will show that~$U$ and~$V$ intersect non-trivially.

According to \cref{contains disk neighbourhood} we may assume that~$V = D_a$ for some radius~$a > 0$.
The point~$x_1 ≔ (a/2, 0)$ does not lie in~$D_1$, and therefore lies in~$C$.
It follows from \cref{contains intersection of ball and disk} that either
\begin{itemize*}

	\item
		there exist a radius~$ε > 0$ with~$x_1 ∈ \ball_ε(x_1) ⊆ U$, or

	\item
		there exist radii~$ε > 0$ and~$a' > 0$ with~$x_1 ∈ \ball_ε(x_1) ∩ D_{a'} ⊆ U$.

\end{itemize*}
But~$x_1$ cannot be contained in~$D_{a'}$ for any radius~$a' > 0$, whence only the first case can occurr.
There hence exists a radius~$ε > 0$ with~$x_1 ∈ \ball_ε(x_1) ⊆ U$.
We may assume that~$ε < a / 2$.
Every point~$(x, y)$ in~$\ball_ε(x_1)$ with~$y ≠ 0$ is then also contained in~$D_a$, whence
\[
	U ∩ V
	⊇
	\ball_ε(x_1) ∩ D_a
	≠
	∅ \,.
\]

\begin{figure}
	\centering
	\begin{tikzpicture}[scale = 2.5]
		% points
		\draw[fill] (0, 0) circle (0.02);
		\draw[fill] (0.5, 0) circle (0.02);
		% D_a
		\draw[pattern = north east lines,
		      pattern color = black]
		      (-1, 0.015)
		      arc (180 : 0 : 1)
		      -- (0, 0.015)
		      -- (0, -0.015)
		      -- (1, -0.015)
		      arc (0 : -180: 1)
		      -- (0, -0.015)
		      -- (0, 0.015)
		      -- cycle;
		% ε-ball
		\draw[pattern = north west lines,
		      pattern color = black]
		      (0.5, 0) circle (0.34);
	\end{tikzpicture}
	\caption{The areas~$D_a$ and~$\ball_ε(x)$ intersect non-trivially.}
\end{figure}
