\subsection{}

For all~$x ∈ ℝ^2$ and~$r > 0$ let
\[
	S_r(x) ≔ \{ y ∈ ℝ^2 \suchthat \abs{x - y} < r \}
\]
be the circle with center~$x$ and radius~$r$.
The four spaces~$A$,~$B$,~$C$ and~$D$ are given by
\[
	A = ℝ / ℤ \,,
	\quad
	B = ⋃_{n ≥ 1} S_{1/n}((0, 1/n)) \,,
	\quad
	C = ⋃_{n ≥ 1} S_n((0, n)) \,,
	\quad
	D = \sphere^1 / N \,.
\]
We will first show that~$A$ is not homeomorphic to either~$B$ or~$C$.
Then we will show that~$B$ and~$C$ are not homeomorphic.
Finally, we will show that~$B$ is homeomorphic to~$D$.



\subsubsection*{$A$ is not homeomorphic to either~$B$ or~$C$}

\begin{claim}
	\label{A is not first countable}
	The space~$A$ is not first countable.
\end{claim}

\begin{proof}
	The space~$A$ arises from the real line~$ℝ$ by collapsing~$ℤ$ to a single point~$x$.
	Let~$p$ be the canonical quotient map from~$ℝ$ to~$ℝ / ℤ$.

	Let~$U_n$ with~$n ∈ ℤ$ be an arbitrary countable collection of open neighbourhoods of~$x$.
	Each preimage~$p^{-1}(U_n)$ is then a saturated open subset of~$ℝ$ that contains all of~$ℤ$.
	This entails that there exists for every integer~$n$ some radius~$ε_n > 0$ such that
	\[
		\ball_{ε_n}(n) ⊆ p^{-1}(U_n) \,.
	\]
	For every integer~$n$ let~$δ_n$ be another radius with~$δ_n < ε_n$ and~$δ_n < 1/3$ (this second condition ensures that the balls of radii~$δ_n$ and~$δ_m$ with~$n ≠ m$ do not intersect).
	The set
	\[
		V ≔ ⋃_{n ∈ ℤ} \ball_{δ_n}(n)
	\]
	is again saturated, open, and contains all of~$ℤ$, whence~$p(V)$ is an open neighbourhood of~$x$.
	This set~$V$ contains none of the balls~$\ball_{ε_n}(n)$, and therefore none of the preimages~$p^{-1}(U_n)$.
	Consequently, no~$U_n$ is contained in~$p(V)$.
	This shows that the sets~$U_n$ with~$n ∈ ℤ$ are not a neighbourhood basis for~$x$.
\end{proof}

The spaces~$B$ and~$C$ are subspaces of the second countable space~$ℝ^2$, therefore again second countable, and thus first-countable.
It follows that~$A$ is not homeomorphic to neither~$B$ nor~$C$.



\subsubsection*{$B$ and~$C$ are not homeomorphic}

\begin{claim}
	The space~$B$ is compact.
\end{claim}

\begin{proof}
	We already know that~$B$ is bounded in~$ℝ^2$, so it suffices to show that~$B$ is closed in~$ℝ^2$.
	To this end we consider for every~$n ≥ 1$ the auxiliary set
	\[
		B_n ≔ ⋃_{k = 1}^{n - 1} S_{1/k}((0, 1/k)) ∪ \cball_{1/n}((0, 1/n)) \,.
	\]
	Graphically speaking, we are only considering the first~$n - 1$ many rings and fill out the~\nth{$n$} ring, turning it into a solid disk.
	Each set~$B_n$ is a union of finitely many closed sets, and therefore again closed.
	We have the identity
	\[
		B = ⋂_{n ≥ 1} B_n \,.
	\]
	Consequently,~$B$ is again closed.
\end{proof}

The subspace~$C$ of~$ℝ^2$ is not bounded, and therefore not compact.
It follows that~$B$ and~$C$ are not homeomorphic.



\subsubsection*{$B$ and~$D$ are homeomorphic}

We will rely on some auxiliary observations.

\begin{lemma}
	\label{removing a point from compact hausdorff}
	Let~$X$ be a compact Hausdorff space and let~$x_0$ be an arbitrary point in~$X$.
	The space~$X - \{ x_0 \}$ is a locally compact Hausdorff space, whose one-point compactification is homeomorphic to~$X$.
\end{lemma}

\begin{proof}
	The set~$X - \{ x_0 \}$ is open in~$X$.
	It follows from Theorem~11.8 that~$X$ is a locally compact Hausdorff space.
	
	Let~$X'$ be the one-point compactification of~$X - \{ x_0 \}$.
	Let~$f$ be the bijection from~$X$ to~$X'$ given by~$f(x) = x$ for~$x ≠ x_0$ and~$f(x_0) = ∞$.
	We claim that~$f$ is a homeomorphism.
	As~$X$ is compact and~$X'$ is a Hausdorff space, it suffices to show that~$f$ is continuous.

	Let~$U$ be an arbitrary open subset of~$X'$.
	The set~$U$ is either an open subset of~$X - \{ x_0 \}$ or the complement of a compact subspace of~$X - \{ x_0 \}$.
	\begin{casedistinction}

		\item
			If~$U$ is an open subset of~$X - \{ x_0 \}$, then~$f^{-1}(U) = U$, whence~$f^{-1}(U)$ is open in~$X - \{ x_0 \}$.
			But~$X - \{ x_0 \}$ is open in~$X$, so it follows that~$f^{-1}(U)$ is also open in~$X$.

		\item
			Otherwise, there exists a compact subspace~$C$ of~$X - \{ x_0 \}$ with
			\[
				U = X' - C \,.
			\]
			We then have the equalities
			\[
				f^{-1}(U)
				=
				f^{-1}(X' - C)
				=
				f^{-1}(X') - f^{-1}(C)
				=
				X - C \,.
			\]
			The space~$C$ is compact and~$X$ is a Hausdorff space, whence it follows that~$C$ is closed in~$X$.
			This tells us that~$f^{-1}(U) = X - C$ is open in~$X$.
		\qedhere

	\end{casedistinction}
\end{proof}

\begin{lemma}
	\label{restriction of identification map}
	Let~$p \colon X \to Y$ be an identification map.
	For every open subset~$V$ of~$Y$, the restriction of~$p$ to a map from~$p^{-1}(V)$ to~$V$ is again an identification map.
\end{lemma}

\begin{proof}
	We denote the restriction of~$p$ to a map from~$p^{-1}(V)$ to~$V$ by~$q$.

	The map~$q$ is surjective and continuous because the original map~$p$ is surjective and continuous.

	Let~$U$ be a subset of~$V$ such that~$q^{-1}(U)$ is open in~$p^{-1}(V)$.
	As~$p^{-1}(V)$ is open in~$X$, it follows that~$q^{-1}(U)$ is also open in~$X$.
	Because~$q^{-1}(U) = p^{-1}(U)$ and~$p$ is an identification map it further follows that~$U$ is open in~$Y$.
	This entails that~$U$ is also open in~$V$.
\end{proof}

\begin{lemma}
	\label{recognizing coproducts}
	Let~$X$ be a topological space and let~$X = ⋃_{i ∈ I}$ be a decomposition into pairwise disjoint open subsets.
	Then~$X$ is homeomorphic to~$∑_{i ∈ I} U_i$.
\end{lemma}

\begin{proof}
	We have for every subset~$V$ of~$X$ the decomposition~$V = ⋃_{i ∈ I} {} (V ∩ U_i)$, with each~$U_i$ being open.
	It follows that~$V$ is open in~$X$ if and only if each intersection~$V ∩ U_i$ is open.
	The topology on~$X$ therefore coincides with the topology of~$∑_{i ∈ I} U_i$.
\end{proof}

\begin{lemma}
	\label{circle without point is homeomorphic to interval}
	Let~$z$ be a point in~$\sphere^1$ with~$z = \eul^{2 \pi \img φ}$ for some~$φ ∈ ℝ$.
	The map
	\[
		(φ, φ + 1) \to \sphere^1 - \{ z \} \,,
		\quad
		α \mapsto \eul^{2 \pi \img α}
	\]
	is a homeomorphism.
\end{lemma}

\begin{proof}
	The proposed map is a continuous bijection, and it is known from real analysis that its inverse is again continuous.
\end{proof}

The space
\[
	B - \{ (0, 0) \}
	=
	⋃_{n ≥ 1} \sphere_{1 / n}( (0, 1/n) ) - \{ (0, 0) \}
\]
is set-theoretically the disjoint union of the arcs
\[
	U_n ≔ \sphere_{1 / n}( (0, 1/n) ) - \{ (0, 0) \} \,.
\]
There exists for every~$x ∈ U_n$ some radius~$ε > 0$ such that the open ball~$\ball_ε(x)$ intersects none of the arcs~$U_m$ with~$m ≠ n$.
Each arc~$U_n$ is therefore open in~$B - \{ (0, 0) \}$.
It follows from \cref{recognizing coproducts} and \cref{circle without point is homeomorphic to interval} that
\[
	B - \{ (0, 0) \}
	≅
	∑_{n ≥ 1} U_n
	≅
	∑_{n ≥ 1} {} (0, 1) \,.
\]

We observe that~$D$ is a compact Hausdorff space:
The space~$D$ is a quotient of~$\sphere^1$, and therefore again compact.
The circle~$\sphere^1$ is normal (because it is a metric space) and its subspace~$N$ is closed.
The quotient~$D$ is therefore again normal by Proposition~13.8.
This entails that it is a Hausdorff space.

Let now~$p$ be the canonical quotient map from~$\sphere^1$ to~$\sphere^1 / N$ and let~$x$ be the image of~$N$ in~$D$.
It follows from \cref{restriction of identification map} that~$p$ restricts to an identification map between~$D - \{ x \}$ and~$\sphere^1 - N$.
This identification map is injective, and therefore a homeomorphism.
It follows from \cref{circle without point is homeomorphic to interval} that
\[
	\sphere^1 - N
	≅
	(0, 1) - \biggl\{ \frac{1}{2^n} \suchthat[\bigg] n ≥ 1 \biggr\} \,,
\]
and from \cref{recognizing coproducts} that
\[
	(0, 1) - \biggl\{ \frac{1}{2^n} \suchthat[\bigg] n ≥ 1 \biggr\}
	≅
	∑_{n ≥ 1} {} \biggl( \frac{1}{2^n}, \frac{1}{2^{n - 1}} \biggr)
	≅
	∑_{n ≥ 1} {} (0, 1) \,.
\]

We have overall seen that
\[
	B - \{ (0, 0) \}
	≅
	∑_{n ≥ 1} {} (0, 1)
	≅
	\sphere^1 - N
	≅
	D - \{ x \} \,,
\]
where both~$B$ and~$D$ are compact Hausdorff spaces.
It follows from \cref{removing a point from compact hausdorff} that already~$B ≅ D$.
