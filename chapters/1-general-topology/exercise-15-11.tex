\subsection{}



\subsubsection*{The case of~$\RProj^n$}

The projective space~$\RProj^n$ has been defined in Example~13.6 as the quotient of~$\sphere^n$ that arises from identifying antipodal points.

\begin{proposition}
	The projective space~$\RProj^n$ is metrizable, and therefore in particular a Hausdorff space.
\end{proposition}

\begin{proof}
	For every point~$x$ in~$\sphere^n$ we can consider the matrix~$x x^{\trans}$, which describes the orthogonal projection onto the one-dimensional space spanned by~$x$.
	Two points in~$\sphere^n$ have the same orthogonal projection if and only if they are linearly dependent, which is the case if and only if they are either equal or antipodal.
	The continuous mapping
	\[
		\sphere^n \to \Mat(n, ℝ) \,,
		\quad
		x \mapsto x x^{\trans}
	\]
	therefore induces a continuos injection
	\[
		\RProj^n \to \Mat(n, ℝ) \,,
		\quad
		\class{x} \mapsto x x^{\trans} \,.
	\]
	The quotient~$\RProj^n$ of~$\sphere^n$ is again compact, and the matrix space~$\Mat(n, ℝ)$ is metrizable and thus a Hausdorff space.
	The above injection is therefore a topological embedding, i.e., a homeomorphism into its image.
\end{proof}

The orthogonal group~$\Ort(n + 1)$ acts on the sphere~$\sphere^n$ via
\[
	A \act x = A x
	\qquad
	\text{for all~$A ∈ \Ort(n + 1)$,~$x ∈ \sphere^n$} \,.
\]
This action descends to a group-theoretic action of~$\Ort(n + 1)$ on~$\RProj^n$ via
\[
	A \act \class{x} = \class{A x}
	\quad
	\text{for all~$A ∈ \Ort(n + 1)$,~$\class{x} ∈ \RProj^n$} \,.
\]
(The action map~$\Ort(n + 1) × \RProj^n \to \RProj^n$ is again continuous, but we won’t need this fact, and will therefore not prove it.)
The action of~$\Ort(n + 1)$ on~$\sphere^n$ is transitive, whence the induced action on~$\RProj^n$ is again transitive.

We consider the standard basis vector~$e_{n + 1}$.
The map
\[
	f'
	\colon
	\Ort(n + 1) \to \RProj^n \,,
	\quad
	A \mapsto A \act \class{e_{n + 1}}
\]
is continuous since it is the composite of the continuous mappings~$A \mapsto A \act x$ and~$x \mapsto \class{x}$.
The map~$f'$ is surjective because the action of~$\Ort(n + 1)$ on~$\RProj^n$ is transitive.
It follows for the stabilizer group~$H$ of~$\class{e_{n + 1}}$ that~$f'$ induces a continuous bijection
\[
	f
	\colon
	\Ort(n + 1) / H \to \RProj^n \,.
\]
The quotient space~$\Ort(n + 1) / H$ is again compact, and~$\RProj^n$ is a Hausdorff space, whence~$f$ is already a homeomorphism.

We claim that the stabilizer group~$H$ is given by~$\Ort(n) × \Ort(1)$.
Indeed, we have~$A \act \class{e_{n + 1}} = \class{e_{n + 1}}$ if and only if~$A e_{n + 1} = ± e_{n + 1}$.
This is equivalent to~$A$ being of the form
\begin{equation}
	\label{block structure for stabilizer matrix}
	A
	=
	\begin{bmatrix}
		A' & 0   \\
		0  & ± 1
	\end{bmatrix}
\end{equation}
for some matrix~$A' ∈ \Ort(n)$.
And this means precisely that~$A ∈ \Ort(n) × \Ort(1)$.



\subsubsection*{The case of~$\CProj^n$}

The book hasn’t provided a definition of~$\CProj^n$.
For our purposes, we will define it as the quotient of the unit sphere in~$ℂ^{n + 1}$ by the action of~$ℂ^×$.
We can then copy our above approach:
we replace~$\sphere^n$ by the unit sphere in~$ℂ^{n + 1}$, and~\eqref{block structure for stabilizer matrix} becomes
\[
	A
	=
	\begin{bmatrix}
		A' & 0 \\
		0  & λ
	\end{bmatrix}
\]
with~$A' ∈ \Uni(n)$ and~$λ ∈ \sphere^1$ instead.
