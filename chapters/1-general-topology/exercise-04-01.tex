\subsection{}

We may assume that~$\closure{A} = X$, so that~$A$ is dense in~$X$.

Let~$C_1$ and~$C_2$ be two closed, disjoint subsets of~$B$ and whose union is all of~$B$.
Then there exist two closed subsets~$C'_1$ and~$C'_2$ of~$X$ with~$C_1 = C'_1 ∩ B$ and~$C_2 = C'_2 ∩ B$.
The sets~$C'_1 ∩ A = C_1 ∩ A$ and~$C'_2 ∩ A = C_2 ∩ A$ are closed in~$A$, disjoint, and their union is all of~$A$.
It follows that~$A = C'_1 ∩ A$ or~$A = C'_2 ∩ A$ because~$A$ is connected.
We may assume that~$A = C'_1 ∩ A$.
This means that~$A ⊆ C'_1$.
As~$A$ is dense in~$X$, it follows that~$C'_1 = X$.
It further follows that~$C_1 = C'_1 ∩ B = X ∩ B = B$.
This shows that~$B$ is connected.

In the special case of~$B = X$ we can also give another proof.
Let~$d \colon X \to D$ be a discrete-valued map with~$D$ nonempty.
The restriction~$\restrict{d}{A}$ is again discrete-valued, and thus constant because~$A$ is connected.
Let~$c$ be the constant value of~$\restrict{d}{A}$.
The preimage~$d^{-1}(c)$ is a closed subset of~$X$ containing the dense subset~$A$ of~$X$, whence~$d^{-1}(c) = X$.
This shows that~$d$ is constant.
