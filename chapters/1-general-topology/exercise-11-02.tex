\subsection{}

\begin{proposition}
	\label{stone cech compactification of compact hausdorff space}
	Let~$X$ be a compact Hausdorff space.
	The canonical map~$Φ_X$ from~$X$ into~$β(X)$ is a homeomorphism.
\end{proposition}

\begin{proof}
	Let~$F$ be the set of all continuous maps from~$X$ to~$[0, 1]$.
	The subspace~$Φ_X(X)$ of~$[0, 1]^F$ is again compact, and therefore closed because~$[0, 1]^F$ is a Hausdorff space.
	It follows that
	\[
		β(X) = \closure{Φ_X(X)} = Φ_X(X) \,.
	\]
	The embedding~$Φ_X$ from~$X$ into~$β(X)$ is therefore bijective, and thus a homeomorphism.
\end{proof}

Let more generally~$f \colon X \to Y$ be a continuous map where~$Y$ is a compact Hausdorff space.
We want to show that~$f$ extends uniquely to a continuous map~$g \colon β(X) \to Y$.
In other words, we want to show that there exists a unique continuous map~$g$ that makes the following diagram commute:
\[
	\begin{tikzcd}
		X
		\arrow{r}[above]{f}
		\arrow{d}[left]{Φ_X}
		&
		Y
		\\
		β(X)
		\arrow[dashed]{ur}[below right]{g}
		&
		{}
	\end{tikzcd}
\]

We know from \cref{stone cech compactification of compact hausdorff space} that the canonical map~$Φ_Y$ from~$Y$ into~$β(Y)$ is a homeomorphism.
It therefore suffices to show that~$f$ extends uniquely to a continuous map~$h \colon β(X) \to β(Y)$;
the maps~$g$ and~$h$ are then related via the equation~$h = Φ_Y ∘ g$.
In other words, we need to show that there exists a unique continuous map~$h$ that makes the following diagram commute:
\[
	\begin{tikzcd}
		X
		\arrow{r}[above]{f}
		\arrow{d}[left]{Φ_X}
		&
		Y
		\arrow{d}[right]{Φ_Y}
		\\
		β(X)
		\arrow{r}[above]{h}
		&
		β(Y)
	\end{tikzcd}
\]
We have already proven the existence and uniqueness of~$h$ in the previous exercise (Exercise~I.11.1).
